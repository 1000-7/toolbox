\newpage
\subsection{Attributes}
\label{Attributes:ID}

\begin{description}
\item[Deadline:] M12
\item[Responsible:] Sigve
\item[Package:] \texttt{eu.amidst.core.databases}
\end{description}

%--------------------------------------------------------------------------------------------
\subsubsection*{Description}
%--------------------------------------------------------------------------------------------

Attributes serve as an intermediary to build either static or dynamic variables which are parsed from the input dataset and/or directly specified by the user (in particular using an additional class that extends this class). 

%--------------------------------------------------------------------------------------------
\subsubsection*{Detailed functionality}
%--------------------------------------------------------------------------------------------

\begin{itemize}
\item List of objects of the type Attribute. This list becomes Unmodifiable after construction.
\item There can be two special attributes, namely, ``TIME\_ID" AND ``SEQUENCE\_ID". The former refers to a temporal identifier, whereas the second identifies a particular sequence (e.g., a client in CajaMar or a drill in Verdande). They are only used whenever they explicitly appear in the dataset with that particular names.
\end{itemize}

%--------------------------------------------------------------------------------------------
\subsubsection*{Code example}
%--------------------------------------------------------------------------------------------

\begin{table}[H]
\begin{tabular}{l} \hline
        \texttt{DataOnDisk data = new DynamicDataOnDiskFromFile(new WekaDataFileReader(} \\
        \texttt{new String(``datasets/syntheticDataVerdandeScenario1.arff'')));}\\\\

        \texttt{Attribute attTRQ = data.getAttributes().getAttributeByName(``TRQ'');}\\
        \texttt{Attribute attROP = data.getAttributes().getAttributeByName(``ROP'');}\\\\
        
        \texttt{List$<$Attribute$>$ attributeList = new ArrayList();}\\
        \texttt{attributeList.add(attTRQ);}\\
        \texttt{attributeList.add(attROP);} \\ \hline 

\end{tabular}
\end{table}