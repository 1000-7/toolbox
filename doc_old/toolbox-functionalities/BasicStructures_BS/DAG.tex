\newpage
\subsection{Directed acyclic graph (DAG)}
\label{DAG:ID}

\begin{description}
\item[Deadline:] M12
\item[Responsible:] Hanen
\item[Code-Package:] \texttt{eu.amidst.core.models}
\end{description}

%--------------------------------------------------------------------------------------------
\subsubsection*{Description}
%--------------------------------------------------------------------------------------------

The class Directed acyclic graph (DAG) defines the Bayesian network graphical structure over a list of static variables.

%--------------------------------------------------------------------------------------------
\subsubsection*{Detailed functionality}
%--------------------------------------------------------------------------------------------

\begin{itemize}
\item It defines the parent set for each variable.
\item It test and detect if a DAG contains cycles or not.
\end{itemize}

%--------------------------------------------------------------------------------------------
\subsubsection*{Code example}
%--------------------------------------------------------------------------------------------

\begin{table}[H]
\begin{tabular}{l} \hline

        \texttt{WekaDataFileReader reader = new}\\
         \texttt{~~~~~~WekaDataFileReader("data/dataWeka/contact-lenses.arff");}\\

        \texttt{StaticVariables variables = new StaticVariables(reader.getAttributes());}\\
        \texttt{DAG dag = new DAG(variables);}\\\\

        \texttt{StaticVariables variables = dag.getStaticVariables();}\\
        \texttt{Variable A = variables.getVariableById(0);}\\
        \texttt{Variable B = variables.getVariableById(1);}\\
        \texttt{Variable C = variables.getVariableById(2);}\\
        \texttt{Variable D = variables.getVariableById(3);}\\ 
        \texttt{Variable E = variables.getVariableById(4);}\\ \\         


        \texttt{dag.getParentSet(B).addParent(A);}\\
        \texttt{dag.getParentSet(C).addParent(A);}\\
        \texttt{dag.getParentSet(C).addParent(B);}\\
        \texttt{dag.getParentSet(D).addParent(B);}\\
        \texttt{dag.getParentSet(E).addParent(B);}\\\hline 

\end{tabular}
\end{table}