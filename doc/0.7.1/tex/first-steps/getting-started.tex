\documentclass[10pt,a4paper]{report}
\usepackage[latin1]{inputenc}
\usepackage{amsmath}
\usepackage{amsfonts}
\usepackage{amssymb}
\usepackage{graphicx}

%%% formatting the code
\usepackage{listings}



\usepackage{color}
\lstset{%
	escapeinside={(*}{*)},%
}

\newcommand{\amidstversion}{\input{../../version.txt}}

\lstset{
	frameround=fttt,
%	language=java,
	numbers=left,
	breaklines=true,
	mathescape, 
	columns=fullflexible, 
	basicstyle=\fontfamily{lmvtt}\selectfont,
	keywordstyle=\color{blue}\fontfamily{lmvtt}\selectfont, 
	numberstyle=\color{black}
}
\lstMakeShortInline[columns=fixed]|



\newcommand{\includejavasource}[1]{\lstinputlisting[language=java]{#1}}
\newcommand{\inlinejava}[1]{\lstinline[columns=fixed,language=java]{#1}}

\newcommand{\lang}[1]{}



\usepackage{hyperref}

\begin{document}

\chapter{Getting Started! }\label{getting-started}



\section{Quick start}

Here we explain how to download and run an example project that uses the AMIDST functionality.  You will need to have \textit{java 8, mvn} and \textit{git} installed. For more information, read the \href{requirements.html}{requirements} section.  First, download the example project code:


\begin{verbatim}
$ git clone https://github.com/amidst/example-project.git
\end{verbatim}

Enter in the downloaded folder:
\begin{verbatim}
$ cd example-project/
\end{verbatim}

A code example illustrating the use of the toolbox is provided in the file \textbf{./src/main/java/BasicExample.java}.

Compile and build the package:

\begin{verbatim}
$ mvn clean package
\end{verbatim}

Finally, run the code example previously mentioned:


\begin{verbatim}
$ java -cp target/example-project-full.jar BasicExample
\end{verbatim}

Each time that our model is updated, the following output is shown:

\begin{verbatim}
Processing batch 1:
    Total instance count: 1000
    N Iter: 2, elbo:2781.727395615198
Processing batch 2:
    Total instance count: 2000
    N Iter: 2, elbo:2763.7884038634625

. . .

Processing batch 89:
    Total instance count: 88565
    N Iter: 2, elbo:1524.8632699545146
	
Bayesian Network:
P(codrna_X1 | M, Z0) follows a Normal|Multinomial,Normal
[ alpha = 2.5153648505301884, beta1 = -6.47078042377021, var = 0.012038840802392285 ] | {M = 0}
[ alpha = 0.0, beta1 = 0.0, var = 1.0 ] | {M = 1}

P(codrna_X2 | M, Z0) follows a Normal|Multinomial,Normal
[ alpha = -1.4100844769398433, beta1 = 6.449118564273272, var = 0.09018732085219959 ] | {M = 0}
[ alpha = 0.0, beta1 = 0.0, var = 1.0 ] | {M = 1}

P(codrna_X3 | M, Z0) follows a Normal|Multinomial,Normal
[ alpha = 0.5004820734231348, beta1 = -0.7233270338873005, var = 0.02287282091577493 ] | {M = 0}
[ alpha = 0.0, beta1 = 0.0, var = 1.0 ] | {M = 1}

P(codrna_X4 | M, Z0) follows a Normal|Multinomial,Normal
[ alpha = -3.727658229972866, beta1 = 15.332997451530298, var = 0.035794031399428765 ] | {M = 0}
[ alpha = 0.0, beta1 = 0.0, var = 1.0 ] | {M = 1}

P(codrna_X5 | M, Z0) follows a Normal|Multinomial,Normal
[ alpha = -1.3370521440370204, beta1 = 7.394413026859823, var = 0.028236889224165312 ] | {M = 0}
[ alpha = 0.0, beta1 = 0.0, var = 1.0 ] | {M = 1}

P(codrna_X6 | M, Z0) follows a Normal|Multinomial,Normal
[ alpha = -3.3189931551027154, beta1 = 13.565377369009742, var = 0.007243019620713637 ] | {M = 0}
[ alpha = 0.0, beta1 = 0.0, var = 1.0 ] | {M = 1}

P(codrna_X7 | M, Z0) follows a Normal|Multinomial,Normal
[ alpha = -1.3216192169520564, beta1 = 6.327466251964861, var = 0.01677087665403506 ] | {M = 0}
[ alpha = 0.0, beta1 = 0.0, var = 1.0 ] | {M = 1}

P(codrna_X8 | M, Z0) follows a Normal|Multinomial,Normal
[ alpha = 2.235639811622681, beta1 = -5.927480690695894, var = 0.015383139745907676 ] | {M = 0}
[ alpha = 0.0, beta1 = 0.0, var = 1.0 ] | {M = 1}

P(codrna_Y) follows a Multinomial
[ 0.3333346978625332, 0.6666653021374668 ]
P(M | codrna_Y) follows a Multinomial|Multinomial
[ 0.9999877194382871, 1.2280561712892748E-5 ] | {codrna_Y = 0}
[ 0.9999938596437365, 6.1403562634704065E-6 ] | {codrna_Y = 1}
P(Z0 | codrna_Y) follows a Normal|Multinomial
Normal [ mu = 0.2687114577360176, var = 6.897846922968294E-5 ] | {codrna_Y = 0}
Normal [ mu = 0.2674517087293682, var = 5.872354808764403E-5 ] | {codrna_Y = 1}


P(codrna_Y|codrna_X1=0.7) = [ 0.49982925627218583, 0.5001707437278141 ]



\end{verbatim}


The output shows: the current batch number; the total number of instances that has been processed until now; the required number of iterations for learning from the current batch; and the \textit{elbo (evidence lower bound)}. Finally,  distributions in the learnt Bayesian network are given.

In general, for start using the AMIDST toolbox, add the following lines to the pom.xml file of your maven project:




\begin{lstlisting}
<repositories>
  <repositories>
    <repository>
  <id>amidstRepo</id>
  <url>https://raw.github.com/amidst/toolbox/mvn-repo/</url>
  </repository>
</repositories>

<dependencies>
  <dependency>
    <groupId>eu.amidst</groupId>
    <artifactId>module-all</artifactId>
    <version>(*\amidstversion*)</version>
    <scope>compile</scope>
  </dependency>
</dependencies>	
\end{lstlisting}


\section{Getting started in detail}

Before starting using the AMDIST, you might check that your system fits
the \href{requirements.html}{requirements} of the toolbox.\newline 

Toolbox users (i.e. those interested in simply using the functionality
provided by AMIDST)~ might find useful the following tutorials:

\begin{itemize}
	\item
	\href{remoteDeps.html}{Loading AMIDST dependencies from a remote maven
		repository}.
	\item \href{localDeps.html}{Installing a local AMIDST repository}
	\item \href{copydep.html}{Generating the packages for each module and its
		dependencies (command line).}
\end{itemize}

Additionally, for those developers interested in colaborating to AMIDST toolbox could read the following tutorials:

\begin{itemize}
	\item \href{amidst_team_modifications.html}{Basic steps for contributing}
\end{itemize}



\end{document}