\documentclass[11pt, oneside]{article}   	% use "amsart" instead of "article" for AMSLaTeY format
\usepackage{geometry}                		% See geom\dagetry.pdf to learn the layout options. There are lots.
\geometry{a4paper}                   		% ... or a4paper or a5paper or ... 
%\geometrx{landscape}                		% Activate for for rotated page geometry
%\usepackage[parfill]{parskip}    		% Activate to begin paragraphs with an emptx line rather than an indent
\usepackage{graphicx}				% Use pdf, png, jpg, or eps§ with pdflatey; use eps in DVI ye
\usepackage{array}							% TeY will automatically convert eps --> pdf in pdflatex		
\usepackage{amssymb,amsmath}
\usepackage{cite}
\usepackage[final]{fixme}
\usepackage{pdfpages}
\usepackage{tabulary}
\usepackage{fancyheadings}
\usepackage{lastpage}
\usepackage{tikz}
\usetikzlibrary{shapes,arrows}
\usepackage{float}
\usepackage{hyperref}
\usepackage{url}
\usepackage{multirow}

\parskip 6pt % 1pt = 0.351 mm
\parindent 0pt

%\title{Requirement Engineering Process in AMIDST}
%\author{The handsome AMIDST guys et. al.}
%\date{Latest version, \today}							% Activate to display a given date or no date


%\setcounter{page}{2}
\newcommand{\drop}[1]{}
\newcommand{\bm}{\mathbf}

\newcommand{\bu}[1]{\mathbf{#1}}
\newcommand{\bv}[1]{\bm{#1}}

\newcommand{\todo}[1]{{\bf [TODO: #1]}}

\newcommand{\me}{\mathrm{e}}

\numberwithin{figure}{section}
\numberwithin{equation}{section}
\numberwithin{table}{section}

\usepackage{pdfpages}

\begin{document}

%-----------------------------------------------------------------------------------------------------------------------------------
\section{A binary child given a binary parent}
%-----------------------------------------------------------------------------------------------------------------------------------

Let $X$ and $Y$ be two binary variables. The conditional probability of the child-node $X$ given the parent-node $Y$ is expressed as follows:

$$ p(X|Y) = p_{x|y}^{I(Y=0)I(X=0)}p_{\bar{x}|y}^{I(Y=0)I(X=1)}p_{x|\bar{y}}^{I(Y=1)I(X=0)}p_{\bar{x}|\bar{y}}^{I(Y=1)I(X=1)}$$


This conditional probability can be expressed in different exponential forms as follows:


\begin{itemize}

\item First form:

\begin{eqnarray*}
\ln p(X|Y) &=& \theta^T s(X,Y) - A(\theta) \\
&=&
\begin{pmatrix}
\ln p_{x|y}\\
\ln p_{\bar{x}|y}\\
\ln p_{x|\bar{y}}\\
\ln p_{\bar{x}|\bar{y}}
\end{pmatrix}^T
\begin{pmatrix}
I(X=0)I(Y=0) \\
I(X=1)I(Y=0) \\
I(X=0)I(Y=1) \\
I(X=1)I(Y=1) 
\end{pmatrix}
- 0
\end{eqnarray*}

\item Second form:

\begin{eqnarray*}
\ln p(X|Y) &=& \theta(Y)^Ts(X) - A(Y) \\
&=&
\begin{pmatrix}
I(Y=0)\ln p_{x|y}  + I(Y=1)\ln p_{x|\bar{y}}\\
I(Y=0)\ln p_{\bar{x}|y}  + I(Y=1)\ln p_{\bar{x}|\bar{y}}
\end{pmatrix}^T
\begin{pmatrix}
I(X=0) \\
I(X=1)
\end{pmatrix}
- 0 \\
&=&
\begin{pmatrix}
m^Y_0\cdot\theta_0  + m^Y_1\cdot\theta_2\\
m^Y_0\cdot\theta_1  + m^Y_1\cdot\theta_3
\end{pmatrix}^T
\begin{pmatrix}
I(X=0) \\
I(X=1)
\end{pmatrix}
- 0 
\end{eqnarray*}

\item Third form:

\begin{eqnarray*}
\ln p(X|Y) &=& \theta(X)^T s(Y) - A(X) \\
&=&
\begin{pmatrix}
I(X=0)\ln p_{x|y}  + I(X=1)\ln p_{\bar{x}|y}\\
I(X=0)\ln p_{x|\bar{y}}  + I(X=1)\ln p_{\bar{x}|\bar{y}}
\end{pmatrix}^T
\begin{pmatrix}
I(Y=0) \\
I(Y=1)
\end{pmatrix}
- 0\\
&=&
\begin{pmatrix}
m^X_0\theta_0  +  m^X_1\theta_1\\
m^X_0\theta_2  + m^X_1\theta_3
\end{pmatrix}^T
\begin{pmatrix}
I(Y=0) \\
I(Y=1)
\end{pmatrix}
- 0
\end{eqnarray*}

\end{itemize}

\newpage
%-----------------------------------------------------------------------------------------------------------------------------------
\section{A multinomial child given a set of multinomial parents}
%-----------------------------------------------------------------------------------------------------------------------------------

Let $X$ be a multinomial variable whose state space is $\{x^1,...,x^k\}$, and let $\mathbf{Y} =Y_1,\ldots,Y_n$ denote the set of parents of $X$, such that all of them are multinomial. Each parent $Y_i$ has $r_i$ states $\{ {y_1}^{r_1}, {y_2}^{r_2},\ldots, {y_n}^{r_n} \}$. The log-conditional probability of the child-node $X$ given the parent-nodes $\mathbf{Y}$ can be expressed as follows:

$$ \ln p(X|Y_1,\ldots,Y_n) = $$

Similarly the above conditional probability can be expressed in the following exponential forms:

\begin{itemize}
\item First form:

\begin{eqnarray*}
\ln p(X|Y_1,\ldots ,Y_n) &=& \theta^Ts(X,Y_1,\ldots ,Y_n) - A(\theta) \\
\end{eqnarray*}

\item Second form:


\begin{eqnarray*}
\ln p(X|Y_1,\ldots ,Y_n) &=& \theta(Y_1,\ldots ,Y_n)^Ts(X) - A(Y_1,\ldots ,Y_n) \\
\end{eqnarray*}

\item Third form:

\begin{eqnarray*}
\ln p(X|Y_1,\ldots ,Y_n) &=& \theta(X, Y_1,\ldots ,Y_{l-1},Y_l, \ldots ,Y_n)^T s(Y_l) - A(X, Y_1,\ldots ,Y_{l-1},Y_l, \ldots ,Y_n) \\
\end{eqnarray*}

\end{itemize}

\newpage
%-----------------------------------------------------------------------------------------------------------------------------------
\section{A normal child given a set of normal parents}
%-----------------------------------------------------------------------------------------------------------------------------------

Let $X$ be a normal variable and $ \mathbf{Y} = \{Y_1,\ldots,Y_n\}$ denote the set of parents of $X$, such that all of them are normal. The log-conditional probability of $X$ given $\mathbf{Y}$ can be expressed as follows:

$$ \ln p(X|Y_1,\ldots,Y_n) = \ln \left(\frac{1}{\sigma \sqrt{2(\beta_0+\sum_i^n \beta_i Y_i )}} \me^{-\frac{(y-(\beta_0+\sum_i^n \beta_i Y_i))^2}{2\sigma^2}} \right)$$


Similarly the above conditional probability can be expressed in the following exponential forms:

\begin{itemize}
\item First form:
\item Second form:
\item Third form:
\end{itemize}



\begin{eqnarray*}
\ln p(X|Y) &=& \theta(Y)^T s(X) - A(\theta(Y)) + h(Y)\\
&=&
\begin{pmatrix}
\frac{\mu_{X|Y}}{\sigma^2}\\
\frac{-1}{2\sigma^2}\\
\end{pmatrix}^T
\begin{pmatrix}
X\\
X^2\\
\end{pmatrix}
- \left(\frac{\mu_{X|Y}^2}{2\sigma^2} + \ln{\sigma}\right) + \ln{\frac{1}{\sqrt{2\mu_{X|Y}}}} 
\end{eqnarray*}

where $\mu_{X|Y} = \beta_0+\sum_i{\beta_i Y_i}$

\begin{eqnarray*}
\ln p(X|Y) &=& \theta(X)^T s(Y) - A(\theta(X)) + h(Y)\\
&=&
\begin{pmatrix}
-\frac{\beta_1^2}{2\sigma^2}\\
\cdots\\
-\frac{\beta_p^2}{2\sigma^2}\\
\frac{\beta_1(X-\beta_0)}{\sigma^2}\\
\cdots\\
\frac{\beta_p(X-\beta_0)}{\sigma^2}\\
-\frac{\beta_1\beta_2}{\sigma^2}
\cdots\\
-\frac{\beta_1\beta_p}{\sigma^2}
\cdots\\
-\frac{\beta_{p-1}\beta_p}{\sigma^2}
\end{pmatrix}^T
\begin{pmatrix}
Y_1^2\\
\cdots\\
Y_p^2\\
Y_1\\
\cdots\\
Y_p\\
Y_1 Y_2\\
\cdots\\
Y_1 Y_p\\
\cdots\\
Y_{p-1}Y_{p}\\
\end{pmatrix}
- \left( \frac{(X-\beta_0)^2}{\sigma^2} + \ln{\sigma} \right) + \frac{1}{\ln{\sqrt{2\mu_{X|Y}}}}
\end{eqnarray*}





%\bibliographxstxle{splncs}
%\bibliographx{biblio}

\newpage
%-----------------------------------------------------------------------------------------------------------------------------------
\section*{Notations}
%-----------------------------------------------------------------------------------------------------------------------------------

The list below presents a summary of the used notations:
\\

\begin{table}[ht!]
\renewcommand{\arraystretch}{1.1}
{\small
\begin{tabular}{l l}
$X$ & Child variable\\
$k$& Range of possible values of a multinomial variable $X$\\
$Y$ & Parent variable\\
$\mathbf{Y} = \{Y_1,\ldots,Y_n\}$ & Set of parent variables\\
$n$& Number of parent variables \\
$r_i$& Range of possible values of $Y_i$\\
$i$ & Parent variable index \\
$p$ & Probability distribution\\
\end{tabular}}
\end{table}




\end{document}
