\documentclass[11pt, oneside]{article}   	% use "amsart" instead of "article" for AMSLaTeY format
\usepackage{geometry}                		% See geom\dagetry.pdf to learn the layout options. There are lots.
\geometry{a4paper}                   		% ... or a4paper or a5paper or ... 
%\geometrx{landscape}                		% Activate for for rotated page geometry
%\usepackage[parfill]{parskip}    		% Activate to begin paragraphs with an emptx line rather than an indent
\usepackage{graphicx}				% Use pdf, png, jpg, or eps§ with pdflatey; use eps in DVI ye
\usepackage{array}							% TeY will automatically convert eps --> pdf in pdflatex		
\usepackage{amssymb,amsmath}
\usepackage{cite}
\usepackage[final]{fixme}
\usepackage{pdfpages}
\usepackage{tabulary}
\usepackage{fancyheadings}
\usepackage{lastpage}
\usepackage{tikz}
\usetikzlibrary{shapes,arrows}
\usepackage{float}
\usepackage{hyperref}
\usepackage{url}
\usepackage{multirow}

\parskip 6pt % 1pt = 0.351 mm
\parindent 0pt

%\title{Requirement Engineering Process in AMIDST}
%\author{The handsome AMIDST guys et. al.}
%\date{Latest version, \today}							% Activate to display a given date or no date


%\setcounter{page}{2}
\newcommand{\drop}[1]{}
\newcommand{\bm}{\mathbf}

\newcommand{\bu}[1]{\mathbf{#1}}
\newcommand{\bv}[1]{\bm{#1}}

\newcommand{\todo}[1]{{\bf [TODO: #1]}}

\newcommand{\me}{\mathrm{e}}

\numberwithin{figure}{section}
\numberwithin{equation}{section}
\numberwithin{table}{section}

\usepackage{pdfpages}

\begin{document}

%-----------------------------------------------------------------------------------------------------------------------------------
\section{A binary child given a binary parent}
%-----------------------------------------------------------------------------------------------------------------------------------

Let $X$ and $Y$ be two binary variables. The log-conditional probability of the child-node $X$ given its parent-node $Y$ is expressed as follows:

\begin{eqnarray*}
\ln p(X|Y) =  I(X=0) I(Y=0) \ln p_{x|y} + I(X=1) I(Y=0) \ln p_{\bar{x}|y} \\
+ I(X=0) I(Y=1) \ln p_{x|\bar{y}} + I(X=1) I(Y=1) \ln p_{\bar{x}|\bar{y}}
\end{eqnarray*}

This conditional probability can be expressed in different exponential forms as follows:


\begin{itemize}

\item \textbf{First form}:

\begin{eqnarray*}
\ln p(X \mid Y) &=& \theta^T s(X,Y) - A(\theta) \\
&=&
\begin{pmatrix}
\ln p_{x|y}\\
\ln p_{\bar{x}|y}\\
\ln p_{x|\bar{y}}\\
\ln p_{\bar{x}|\bar{y}}
\end{pmatrix}^T
\begin{pmatrix}
I(X=0)I(Y=0) \\
I(X=1)I(Y=0) \\
I(X=0)I(Y=1) \\
I(X=1)I(Y=1) 
\end{pmatrix}
- 0\\
&=&
\begin{pmatrix}
\theta_0\\
\theta_1\\
\theta_2\\
\theta_3
\end{pmatrix}^T
\begin{pmatrix}
I(X=0)I(Y=0) \\
I(X=1)I(Y=0) \\
I(X=0)I(Y=1) \\
I(X=1)I(Y=1) 
\end{pmatrix}
- 0
\end{eqnarray*}

\item \textbf{Second form}:

\begin{eqnarray*}
\ln p(X \mid Y) &=& \theta(Y)^Ts(X) - A(Y) \\
&=&
\begin{pmatrix}
I(Y=0)\ln p_{x|y}  + I(Y=1)\ln p_{x|\bar{y}}\\
I(Y=0)\ln p_{\bar{x}|y}  + I(Y=1)\ln p_{\bar{x}|\bar{y}}
\end{pmatrix}^T
\begin{pmatrix}
I(X=0) \\
I(X=1)
\end{pmatrix}
- 0 \\
&=&
\begin{pmatrix}
m^Y_0\cdot\theta_0  + m^Y_1\cdot\theta_2\\
m^Y_0\cdot\theta_1  + m^Y_1\cdot\theta_3
\end{pmatrix}^T
\begin{pmatrix}
I(X=0) \\
I(X=1)
\end{pmatrix}
- 0 
\end{eqnarray*}

\item \textbf{Third form}:

\begin{eqnarray*}
\ln p(X \mid Y) &=& \theta(X)^T s(Y) - A(X) \\
&=&
\begin{pmatrix}
I(X=0)\ln p_{x|y}  + I(X=1)\ln p_{\bar{x}|y}\\
I(X=0)\ln p_{x|\bar{y}}  + I(X=1)\ln p_{\bar{x}|\bar{y}}
\end{pmatrix}^T
\begin{pmatrix}
I(Y=0) \\
I(Y=1)
\end{pmatrix}
- 0\\
&=&
\begin{pmatrix}
m^X_0 \cdot \theta_0  +  m^X_1\cdot \theta_1\\
m^X_0 \cdot \theta_2  + m^X_1 \cdot \theta_3
\end{pmatrix}^T
\begin{pmatrix}
I(Y=0) \\
I(Y=1)
\end{pmatrix}
- 0
\end{eqnarray*}

\end{itemize}

\newpage
%-----------------------------------------------------------------------------------------------------------------------------------
\section{A multinomial child given a set of multinomial parents}
%-----------------------------------------------------------------------------------------------------------------------------------

Let $X$ be a multinomial variable with $k$ possible values such that $k \geq 2$, and let $\mathbf{Y} =\{Y_1,\ldots,Y_n\}$ denote the set of parents of $X$, such that all of them are multinomial. Each parent $Y_i$, $1 \geq i \geq n$, has $r_i$ possible values or states such that $r_i \geq 2$. A parental configuration for the child-node $X$ is then a set of $n$ elements $\{Y_1 = y_1^{v}, \ldots, Y_i = y_i^{v},\ldots, Y_n = y_n^{v} \}$ such that $y_i^{v}$ denotes a potential value of variable $Y_i$ such that  $1 \leq v \leq r_i$. Let $q = r_1 \times \ldots \times r_n$ denote the total number of parental configurations, and let $\mathbf{y}^l$ denote the $l^{th}$ parental configuration such that $1 \leq l \leq q$.

The log-conditional probability of the child-node $X$ given its parent-nodes $\mathbf{Y}$ can be expressed as follows:

$$ \ln p(X \mid \mathbf{Y}) = \sum_{j=1}^k \sum_{l=1}^q I(X=x^j) I(\mathbf{Y} =\mathbf{y}^l) \ln p_{x^j  \mid \mathbf{y}^l} $$

Similarly the above log-conditional probability can be expressed in the following exponential forms:

\begin{itemize}

\item \textbf{First form}:

\begin{eqnarray*}
\ln p(X \mid \mathbf{Y}) &=& \theta^T s(X,\mathbf{Y}) - A(\theta) \\\\
&=&
\begin{pmatrix}
\ln p_{x^1\mid \mathbf{y}^1}\\
\vdots \\
\ln p_{x^1\mid \mathbf{y}^q}\\
\vdots \\
\ln p_{x^k\mid \mathbf{y}^1}\\
\vdots \\
\ln p_{x^k\mid \mathbf{y}^q}\\
\end{pmatrix}^T
\begin{pmatrix}
I(X=x^1)I(\mathbf{Y}=\mathbf{y}^1) \\
\vdots \\
I(X=x^1)I(\mathbf{Y}=\mathbf{y}^q)\\
\vdots \\
I(X=x^k)I(\mathbf{Y}=\mathbf{y}^1) \\
\vdots \\
I(X=x^k)I(\mathbf{Y}=\mathbf{y}^q)
\end{pmatrix}
- 0 \\\\
&=&
\begin{pmatrix}
\theta_{11}\\
\vdots \\
\theta_{1q}\\
\vdots \\
\theta_{k1}\\
\vdots \\
\theta_{kq}\\
\end{pmatrix}^T
\begin{pmatrix}
I(X=x^1) I(\mathbf{Y}=\mathbf{y}^1) \\
\vdots \\
I(X=x^1) I(\mathbf{Y}=\mathbf{y}^q)\\
\vdots \\
I(X=x^k) I(\mathbf{Y}=\mathbf{y}^1) \\
\vdots \\
I(X=x^k) I(\mathbf{Y}=\mathbf{y}^q)
\end{pmatrix}
- 0
\end{eqnarray*}

\vspace{0.5in}
\item \textbf{Second form}:

\begin{eqnarray*}
\ln p(X \mid \mathbf{Y}) &=& \theta(\mathbf{Y})^Ts(X) - A(\mathbf{Y}) \\ \\
&=&
\begin{pmatrix}
I(\mathbf{Y}=\mathbf{y}^1) \ln p_{x^1\mid \mathbf{y}^1} + \ldots + I(\mathbf{Y}=\mathbf{y}^q)\ln p_{x^1\mid \mathbf{y}^q}\\
\vdots \\
I(\mathbf{Y}=\mathbf{y}^1) \ln p_{x^k\mid \mathbf{y}^1} + \ldots + I(\mathbf{Y}=\mathbf{y}^q)\ln p_{x^k\mid \mathbf{y}^q}\\
\end{pmatrix}^T
\begin{pmatrix}
I(X=x^1) \\
\vdots \\
I(X=x^k) 
\end{pmatrix}
- 0 \\ \\
&=&
\begin{pmatrix}
\mathbf{m}^{\mathbf{Y}}_1 \cdot \theta_{11}  + m^{\mathbf{Y}}_q \cdot \theta_{1q} \\
\vdots \\
\mathbf{m}^{\mathbf{Y}}_1 \cdot \theta_{k1}  + m^{\mathbf{Y}}_q \cdot \theta_{kq}
\end{pmatrix}^T
\begin{pmatrix}
I(X=x^1) \\
\vdots \\
I(X=x^k)
\end{pmatrix}
- 0 
\end{eqnarray*}

\noindent such that $\mathbf{m}^{\mathbf{Y}}_1 = \prod_{i=1}^n I( Y_i = y_i^1) = \prod_{i=1}^n m^{Y_i}_1$ denotes the expected sufficient statistics for the first parental configuration, and $\mathbf{m}^{\mathbf{Y}}_q = \prod_{i=1}^n I( Y_i = y_i^{r_i})  = \prod_{i=1}^n m^{Y_i}_{r_i} $ denotes the expected sufficient statistics for the last parental configuration.

\vspace{0.5in}
\item \textbf{Third form}:

\begin{eqnarray*}
\ln p(X\mid \mathbf{Y}) &=& \theta(X)^T s(\mathbf{Y}) - A(X) \\ \\
&=&
\begin{pmatrix}
I(X=x^1)  \ln p_{x^1\mid \mathbf{y}^1}  + \ldots + I(X=x^k)  \ln p_{x^k\mid \mathbf{y}^1} \\
\vdots \\
I(X=x^1)  \ln p_{x^1\mid \mathbf{y}^q}  + \ldots + I(X=x^k)  \ln p_{x^k\mid \mathbf{y}^q}
\end{pmatrix}^T
\begin{pmatrix}
I(\mathbf{Y}=\mathbf{y}^1) \\
\vdots \\
I(\mathbf{Y}=\mathbf{y}^q)
\end{pmatrix}
- 0\\ \\
&=&
\begin{pmatrix}
m^X_1 \cdot \theta_{11}  +  \ldots + m^X_k \cdot \theta_{k1}\\
\vdots \\
m^X_1 \cdot \theta_{1q}   + \ldots + m^X_k \cdot \theta_{kq}
\end{pmatrix}^T
\begin{pmatrix}
I(\mathbf{Y}=\mathbf{y}^1) \\
\vdots \\
I(\mathbf{Y}=\mathbf{y}^q)
\end{pmatrix}
- 0
\end{eqnarray*}

\end{itemize}


\newpage
%-----------------------------------------------------------------------------------------------------------------------------------
\section{A normal child given a set of normal parents}
%-----------------------------------------------------------------------------------------------------------------------------------

Let $X$ be a normal variable and $ \mathbf{Y} = \{Y_1,\ldots,Y_n\}$ denote the set of parents of $X$, such that all of them are normal. 

The log-conditional probability of $X$ given its parents $\mathbf{Y}$ can be expressed as follows:

$$ \ln p(X|Y_1,\ldots,Y_n) = \ln \left(\frac{1}{\sigma \sqrt{2(\beta_0+\sum_i^n \beta_i Y_i )}} \me^{-\frac{(y-(\beta_0+\sum_i^n \beta_i Y_i))^2}{2\sigma^2}} \right)$$


Similarly the above log-conditional probability can be expressed in the following exponential forms:

\begin{itemize}
\item \textbf{First form}:

\begin{eqnarray*}
\ln p(X \mid \mathbf{Y}) &=& \theta^T s(X,\mathbf{Y}) - A(\theta) + h(\mathbf{Y})\\\\
&=&
\begin{pmatrix}
\frac{-1}{2\sigma^2} &=& \theta_{\mbox{-}1} \\
\frac{-\beta_1^2}{2\sigma^2} &=& \theta_{1^2} \\
\vdots \\
\frac{-\beta_n^2}{2\sigma^2} &=& \theta_{n^2} \\
\frac{\beta_0}{\sigma^2} &=& \theta_0 \\
\frac{\beta_1}{\sigma^2} &=& \theta_1 \\
\vdots\\
\frac{\beta_n}{\sigma^2} &=& \theta_n \\\\
\frac{-\beta_0\beta_1}{\sigma^2} &=&\theta_{01} \\
\vdots\\
\frac{-\beta_0\beta_n}{\sigma^2} &=& \theta_{0n} \\
\frac{-\beta_1\beta_2}{\sigma^2} &=& \theta_{12} \\
\vdots\\
\frac{-\beta_1\beta_n}{\sigma^2} &=& \theta_{1n} \\
\vdots\\
\frac{-\beta_{n-1}\beta_n}{\sigma^2} &=&\theta_{n\mbox{-}1n} \\
\end{pmatrix}^T
\begin{pmatrix}
X^2\\
Y^2\\
\vdots\\
Y_n^2\\
X\\
XY_1\\
\vdots\\
XY_n\\
Y_1\\
\vdots\\
Y_n\\
Y_1Y_2\\
\vdots\\
Y_1Y_n\\
\vdots\\
Y_{n-1}Y_n
\end{pmatrix}
- \left( \frac{\beta_0^2}{2\sigma^2} + \ln{\sigma}\right) + \frac{1}{\ln{\sqrt{2\mu_{X|Y}}}}
\end{eqnarray*}

\vspace{0.5in}
\item \textbf{Second form}:

\begin{eqnarray*}
\ln p(X\mid \mathbf{Y}) &=& \theta(\mathbf{Y})^T s(X) - A \big(\theta(\mathbf{Y})\big) + h(\mathbf{Y})\\\\
&=&
\begin{pmatrix}
\frac{\mu_{X|Y}}{\sigma^2}\\
\frac{-1}{2\sigma^2}\\
\end{pmatrix}^T
\begin{pmatrix}
X\\
X^2\\
\end{pmatrix}
- \left(\frac{\mu_{X|Y}^2}{2\sigma^2} + \ln{\sigma}\right) + \ln{\frac{1}{\sqrt{2\mu_{X|Y}}}} \\\\
&=&
\begin{pmatrix}
\theta_0+\theta_i m_0^{Y_i}\\
\theta_{\mbox{-}1}\\
\end{pmatrix}^T
\begin{pmatrix}
X\\
X^2\\
\end{pmatrix}
- \left(\frac{\mu_{X|Y}^2}{2\sigma^2} + \ln{\sigma}\right) + \ln{\frac{1}{\sqrt{2\mu_{X|Y}}}} 
\end{eqnarray*}

where $\mu_{X|Y} = \beta_0+\sum_i^n{\beta_i Y_i}$

\vspace{0.2in}
\item \textbf{Third form}:

\begin{eqnarray*}
\ln p(X\mid \mathbf{Y}) &=& \theta(X)^T s(\mathbf{Y}) - A \big(\theta(X) \big) + h(\mathbf{Y})\\\\
&=&
\begin{pmatrix}
-\frac{\beta_1^2}{2\sigma^2}\\
\cdots\\
-\frac{\beta_n^2}{2\sigma^2}\\
\frac{\beta_1(X-\beta_0)}{\sigma^2}\\
\cdots\\
\frac{\beta_n(X-\beta_0)}{\sigma^2}\\
-\frac{\beta_1\beta_2}{\sigma^2}\\
\cdots\\
-\frac{\beta_1\beta_n}{\sigma^2}\\
\cdots\\
-\frac{\beta_{n-1}\beta_n}{\sigma^2}
\end{pmatrix}^T
\begin{pmatrix}
Y_1^2\\
\cdots\\
Y_n^2\\
Y_1\\
\cdots\\
Y_n\\
Y_1 Y_2\\
\cdots\\
Y_1 Y_n\\
\cdots\\
Y_{n-1}Y_{n}\\
\end{pmatrix}
- \left( \frac{(X-\beta_0)^2}{\sigma^2} + \ln{\sigma} \right) + \frac{1}{\ln{\sqrt{2\mu_{X|Y}}}}\\\\
&=&
\begin{pmatrix}
\theta_{1^2}\\
\cdots\\
\theta_{n^2}\\
\theta_1 m_0^X+\theta_{01}\\
\cdots\\
\theta_n m_0^X+\theta_{0n}\\
\theta_{12}\\
\cdots\\
\theta_{1n}\\
\cdots\\
\theta_{n\mbox{-}1n}
\end{pmatrix}^T
\begin{pmatrix}
Y_1^2\\
\cdots\\
Y_n^2\\
Y_1\\
\cdots\\
Y_n\\
Y_1 Y_2\\
\cdots\\
Y_1 Y_n\\
\cdots\\
Y_{n-1}Y_{n}\\
\end{pmatrix}
- \left( \frac{(X-\beta_0)^2}{\sigma^2} + \ln{\sigma} \right) + \frac{1}{\ln{\sqrt{2\mu_{X|Y}}}}
\end{eqnarray*}

\end{itemize}


\newpage
%-----------------------------------------------------------------------------------------------------------------------------------
\section*{Notations}
%-----------------------------------------------------------------------------------------------------------------------------------

The list below presents a summary of the used notations:
\\

\begin{table}[ht!]
\renewcommand{\arraystretch}{1.1}
{\small
\begin{tabular}{l l}
$X$ & Child variable\\
$k$& Range of possible values of a multinomial variable $X$\\
$j$ & Index over $X$ values, i.e., $1 \geq j \geq k$ \\
$Y$ & One parent variable\\
$\mathbf{Y}$ & Set of parent variables\\
$n$& Number of parent variables \\
$i$ & Index over parent variables, i.e., $1 \geq i \geq n$ \\
$r_i$& Range of possible values of a multinomial variable $Y_i$\\
$q $ & Total number of configurations of a multinomial parent set $\mathbf{Y}$\\
$l$ & Index over the possible parental configuration values, i.e., $1 \geq l \geq q$ \\
$\mathbf{y}^l$ & The $l^{th}$ configuration of a multinomial parent set $\mathbf{Y}$\\
$\theta_{jl}$ & Equal to $\ln p_{x^j\mid \mathbf{y}^l}$, denoting the log-conditional probability of $X$ in its state $j$ \\
                    & given the $l^{th}$ parent configuration\\
$p$ & Probability distribution\\
$m$ & Expected sufficient statistics \\
$s$ & Sufficient statistics \\
\end{tabular}}
\end{table}

\end{document}
