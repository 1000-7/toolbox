% !TEX root = D41-report.tex
\section{Introduction}

%\comment{This is just the first  page from D3.1, then quickly terminated after learning is introduced. Just to say what a BN is.}


Probabilistic graphical models provide a well-founded and principled approach for performing inference 
in complex domains endowed with uncertainty. A probabilistic graphical model is a framework consisting 
of two parts: a qualitative component in the form of a graphical model encoding conditional independence 
assertions about the domain being modelled as well as a quantitative component consisting of a collection of local 
probability distributions adhering to the independence properties specified in the graphical model. Collectively, the 
two components provide a compact representation of the joint probability distribution over the domain being modelled. 

Bayesian networks (BNs) \cite{Pearl88} are a particular type of
probabilistic graphical model that has enjoyed widespread attention in
the last two decades. Figure~\ref{fig:sampleBN} shows a BN representing
the joint distribution of variables $X_1,\ldots,X_5$. Attached to each node, there is a conditional probability
distribution given its parents in the network, so that the joint distribution factorises as

\[
p(X_1,\ldots,X_5) = p(X_1) p(X_2|X_1) p(X_3|X_1) p(X_4|X_2,X_3) p(X_5|X_3).
\]

In general, for a BN with $n$ variables $\bX=\{X_1,\ldots,X_n\}$, the joint distribution factorises as

\begin{equation}
\label{equ:factorisation}
p(\bX) = \prod_{i=1}^n p(X_i|\pa{X_i}) ,
\end{equation}
where $\pa{X_i}$ denotes the set of parents of $X_i$ in the network. 


\newcommand{\simpleModel}{    
      \node[obs] (X1) {$X_1$};
      \node[obs] (X2) [below left of=X1, xshift=-1.2cm, yshift=-1.2cm] {$X_2$};
      \node[obs] (X3) [below right of=X1, xshift=+1.2cm, yshift=-1.2cm] {$X_3$};
      \node[obs] (X4) [below right of=X2, xshift=+1.2cm, yshift=-1.2cm] {$X_4$};
      \node[obs] (X5) [below right of=X3, xshift=+1.2cm, yshift=-1.2cm] {$X_5$};
      \edge{X1}{X2};
      \edge{X1}{X3};
      \edge{X2,X3}{X4};
      \edge{X3}{X5};
    }
    
\begin{figure}[htb]
  \begin{center}
   \scalebox{1}{
    \begin{tikzpicture}

    \simpleModel
    \end{tikzpicture}
    }
  \end{center}
  \caption{A Bayesian network with five variables.}
  \label{fig:sampleBN}
\end{figure}


We will use lowercase letters to refer to
values or configurations of values, so that $x$ denotes a value of $X$ and $\bx$ is a configuration of the
variables in $\bX$. 
Given a set of observed variables $\bX_E\subset \bX$ and a set of variables of interest $\bX_I\subset \bX \setminus \bX_E$,
\emph{probabilistic inference} is the calculation  of the posterior distribution
$p(x_i|\bx_E)$ for each $i\in I$. 
A thorough introduction to the state of the art for inference in Bayesian networks was given in \cite{AMIDST-D31}. 

In this document we assume that inference techniques for a given Bayesian network is available, and will consider how to define a Bayesian network model that fits a data set as well as possible.
This process is known as \textit{learning} in the Bayesian network community.



