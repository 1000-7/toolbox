\section{Ongoing and future work}\label{section:conclusions}

Is is important, at this stage, to emphasize that this is an ongoing project research and a better understanding of the faced problems and the feedback provided from the use case providers might reveal that some elements of this modelling framework need to be refined.  For example, the modelling framework is now assuming a first-order Markov assumption. One could easily envision the possibility of going to higher order Markov assumptions, if we find that this first-order assumption severely limits the modelling capacities of this framework in some of the application scenarios. 

As already mentioned above,  we are now working on inference (WP3) and learning algorithms (WP4) that make this modelling framework applicable to scenarios dealing with massive data streams. These developments might also have an impact in the current framework. Special emphasis will be given to the study of bounded and unbounded time horizon models as way to trade-off between model expressibility and model complexity. This kind of analysis will be considered in combination with the proposed inference and learning algorithms.

Finally, we plan to study which are the limitations of the final proposed modelling framework. The focus of this  framework on massive data streams will necessarily limit its scope. A clear identification of these limitations will be also beneficial for the understanding of the chosen trade-offs between expressibility and computational complexity.

The outcome of all these analyses will be included in deliverable 2.2.
