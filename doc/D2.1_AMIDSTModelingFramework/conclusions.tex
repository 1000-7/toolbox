\section{Ongoing and future work}\label{section:conclusions}

Is is important, at this stage, to emphasize that this is an ongoing project research and a better understanding of the faced problems and the feedback provided from the use case providers, might reveal that some elements of this modelling framework need to be refined.  For example, the modelling framework is now assuming a first-order Markov assumption. One could easily envision the possibility of going to higher order Markov assumptions, if we find that this first-order assumption severely limits the modelling capacities of this framework in some of the application scenarios. 

We plan indeed to study which are the limitations of the final proposed modelling framework. This framework needs to handle extremely large data streams, and hence, it should be efficient enough to do that, possibly at the expense of a certain loss of expressibility power. A clear identification of the framework's strengths and limitations will be beneficial for a better understanding of the chosen trade-offs between expressibility and computational complexity.

Finally, as already mentioned above,  we are now working on inference (WP3) and learning algorithms (WP4) that make this modelling framework applicable to scenarios dealing with massive data streams. These developments might also have an impact in the current framework. Special emphasis will be given to the study of bounded and unbounded time horizon models as way to trade-off between model expressibility and model complexity. This kind of analysis will be considered in combination with the proposed inference and learning algorithms.



The outcome of all these analyses will be included in Deliverable 2.2.
