\section{Executive summary}\label{section:executiveSummary}

The main objective of this document is to design a probabilistic model class (AMIDST modelling framework) that can efficiently support massive data streams, and in particular, address the different application scenarios encountered in each use-case provider, namely, Daimler AG, Cajamar, and Verdande Technology AS. 

Given the dynamic nature of the applications as well as the data provided by the different use cases, this document not only includes a static modelling framework proposal, but also, and more importantly, a dynamic modelling framework proposal that extends the static one. The advantage of considering the dynamic models at this early stage is that all the relevant aspects that mainly arise form a dynamic context could be taken into account.  

To this end, we first identify and analyse the family of probability distributions of the data sets provided by each use-case. To ensure a clear understanding of this data analysis and the different proposed models, we also provide a background section including required concepts about data analysis techniques, Bayesian networks, dynamic modelling structures, as well as a touch upon the state-of-the-art of modelling hybrid distributions (which will be more technically reviewed in Deliverables 3.1 and 4.2).

Next, given the provided expert knowledge, the requirement analysis and the performed data analysis, we propose preliminary models for each use-case provider.

Finally, using the combination of the three use-case-tailored static and dynamic models, we define a preliminary three-layered AMIDST modelling framework. In this way, we claim that this framework is able to tackle all use-cases with different data-driven learning methods, as well as to take advantage of structural adjustments with respect to a particular domain.

It is crucial here to note that the current AMIDST modelling framework should be considered as an initial proposal. The proposal might be updated/extended in the future in the light of evaluation results of the proposed model. The final AMIDST modelling framework will be presented in Deliverable 2.2.


%This document describes the AMIDST modelling framework which consists of the general AMIDST model class as well as the specification of the model classes of the three use-case providers, namely, Daimler AG, Cajamar, and Verdande Technology AS.

%First, the identified preliminary model classes are introduced, for each use-case, based on the requirements analysis (see Deliverable 1.2 \cite{Fer14b}) as well as the preliminary data analysis (including the use of several tools such as sample correlograms, sample partial correlograms, histograms and bivariate contour plots).

%Next, the general AMIDST model class is defined such that it ensures three main requirements: 1) it takes all the preliminary model class characteristics into account; 2) it is applicable to any future, potentially similar, use-cases; and 3) it is scalable, supporting both inference and learning from massive data streams.

%It is crucial here to note that the current AMIDST modelling framework should be considered as an initial proposal. The proposal might be updated/extended in the future, in order to better meet AMIDST requirements and be adapted based on new domain insights. The final AMIDST modelling framework will be presented in Deliverable 2.2.
