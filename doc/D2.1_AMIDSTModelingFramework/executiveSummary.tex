\section{Executive summary}\label{section:executiveSummary}

The main objective of this document is to design a probabilistic model class (AMIDST modelling framework) that addresses the different application scenarios encountered in each use-case provider, namely, Daimler AG, Cajamar, and Verdande Technology AS.
% but also another future similar problems dealing with massive data streams.

Given the dynamic nature of the applications, as well as the data and the feedback provided by the different use cases, we found convenient to include in this document not only a static modelling framework proposal, but also, and more importantly, a dynamic modelling framework proposal that extends the static one. By considering the dynamic approach at this early stage, we can start now to evaluate how the modelling hypothesis behind this framework affects the inference and learning algorithms which are being studied in WP3 and WP4, respectively. 
%And, hence, to refine the final modelling framework before it is presented in Deliverable 2.2.

To ensure a clear understanding of the presented framework, we provide in this document a background section including required concepts about data analysis techniques, Bayesian networks and dynamic modelling. We also touch upon the state-of-the-art of modelling complex empirical distributions (which is reviewed in more depth in Deliverables 3.1 and 4.2).

The application scenarios identified in Deliverable 1.2 are addressed here. We firstly start giving an overview about which is the state-of-the-art for the use case providers, i.e. how they currently solve these applications scenarios. We then present the specific proposed models for each application scenario. These models are usually presented as a dynamic extension of a first static proposal. To design these models, we try to apply existing and well-studied probabilistic modelling techniques when possible, but we also introduce new elements when further modelling expressive capacity is needed. Our proposed models, at this stage, are mainly supported by the provided domain/expert knowledge and by a data analysis about the nature and the temporal structure of the distribution families governing the use cases. This data analysis is also used to validate, as much as possible, the elicited knowledge, which due to the complexity of the modelled systems/processes, sometimes is not entirely comprehensive or accurate.

%We also consider previously developed models by some of the use case providers as a starting point. 

Finally, based on the commonalities among the three use-case-tailored models, we introduce the AMIDST modelling framework, which is presented as a dynamic extension of a firstly introduced static framework. We claim that this framework is able to tackle all use-cases. Moreover, the presented framework aims to ease the integration of elicited domain/expert knowledge in combination with data-driving learning methods when applied to other future similar application scenarios. 

%We claim that this framework is able to tackle all use-cases and similar future problems. This framework is %designed over the basis of  by integrating the 
%different data-driven learning methods, as well as to take advantage of structural adjustments with respect to each particular domain.

It is crucial here to note that the current AMIDST modelling framework should be considered as an initial proposal. This proposal might be updated/extended in the future in the light of a better understanding of the faced problems and new feedback provided from the use case providers, as well as the impact of the study of the inference and learning techniques for dealing with massive data streams. The final AMIDST modelling framework will be presented in Deliverable 2.2.


