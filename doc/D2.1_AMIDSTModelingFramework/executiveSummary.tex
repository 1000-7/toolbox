\section{Executive summary}\label{section:executiveSummary}

The main objective of this document is to provide an initial design of a probabilistic model class (the AMIDST modelling
framework) that supports the different application scenarios described for the three use-case providers: Daimler AG, Cajamar, and Verdande Technology AS.
% but also another future similar problems dealing with massive data streams.


% Given the nature of the application scenarios, as well as the data and the feedback provided by the different use cases,
% we found convenient to include in this document not only a static modelling framework proposal, but also, and more
% importantly, a dynamic modelling framework proposal that extends the static one. By considering the dynamic approach at
% this early stage, we can start now to evaluate how the modelling hypothesis behind this framework affects the inference
% and learning algorithms which are being studied in WP3 and WP4, respectively. 

Given the nature of the application scenarios, initial investigations revealed the need for not only considering static
model classes, but also model classes that explicitly capture the  dynamics in the application scenarios. Initial considerations on
dynamic modelling components are therefore also included in this document, although the description of work (DoW)
lists these activities for the ensuing Deliverable 2.2. 

In order to establish a common frame of reference for the proposed framework, we provide a background section including
concepts about data analysis methods, Bayesian networks, and dynamic modelling techniques. We also touch upon
state-of-the-art of modelling complex empirical distributions, although this will be reviewed in more depth in Deliverables
3.1 and 4.2. 

As basis for the development of the general AMIDST modelling framework, we establish preliminary models for the application scenarios identified in
Deliverable 1.2. Specifically, we first start by giving an overview of state-of-the-art for each use-case provider, i.e., how the
application scenarios are currently being addressed. We then present preliminary candidate model classes for each
application scenario. These model classes will typically be presented as dynamic extensions of an initial static model class. At
this stage, the proposed models are mainly supported by  domain/expert knowledge elicited from the use-case providers
as well as by analyses of the nature and temporal structure of the distribution families governing the use cases; the data
analysis is also used to corroborate the elicited domain knowledge.

%We also consider previously developed models by some of the use case providers as a starting point. 

Finally, based on the commonalities among the three use-case-tailored model classes, we introduce the general AMIDST
modelling framework and demonstrate that the framework is able to support the identified use cases. The framework is
furthermore intended to allow for a combination of data-driving learning and integration of domain/expert knowledge. This
is partly achieved by defining the framework based on a single coherent probabilistic model class.


%%% WHAT TO DO WITH THIS::::, and aims to ease the integration of elicited domain/expert knowledge in combination with data-driving learning methods when applied to other future similar application scenarios. 

%We claim that this framework is able to tackle all use-cases and similar future problems. This framework is %designed over the basis of  by integrating the 
%different data-driven learning methods, as well as to take advantage of structural adjustments with respect to each particular domain.

It is crucial here to note that the AMIDST modelling framework described in the present document should be considered as an initial proposal. This
proposal might be updated/extended in the future based on feedback provided from the use-case providers as well as insights
revealed by the inference and learning techniques and results obtained in the later work packages. The final AMIDST
modelling framework will be presented in Deliverable 2.2. 


