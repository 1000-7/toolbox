\section{Executive summary}\label{section:executiveSummary}

The main objective of this document is to design a probabilistic model class (AMIDST modelling framework) that can efficiently support massive data streams, and in particular, address the different application scenarios encountered in each use-case provider, namely, Daimler AG, Cajamar, and Verdande Technology AS. 

Given the dynamic nature of the applications as well as the data provided by the different use cases, this document not only includes a static modelling framework proposal, but also, and more importantly, a dynamic modelling framework proposal that extends the static one. The advantage of considering the dynamic models at this early stage is that all the relevant aspects that mainly arise form a dynamic context could be taken into account. Moreover, we can also start now to evaluate how the modelling hypothesis behind this framework affects the inference and learning algorithms which are being studied in WP3 and WP4. 
%So, we have more room for revising and updating this modelling framework before the final version is proposed in Deliverable 2.2. 

%To this end, we first identify and analyse the family of probability distributions of the data sets provided by %each use-case. 

To ensure a clear understanding of the presented models, we  provide a background section including required concepts about data analysis techniques, Bayesian networks, dynamic modelling, as well as a touch upon the state-of-the-art of modelling complex empirical distributions (which is reviewed in more depth in deliverables D3.1 and 4.2).

The different application scenarios identified in deliverable 1.2 are addressed here. We firstly start giving an overview about which is the state-of-the-art for the use case providers, i.e. how they currently solve these applications scenarios, if this is the case. We then present the specific proposed models which approach each application scenario. These models are usually presented as a dynamic extension of a first static proposal. To design these models we try to apply existing and well-studied probabilistic modelling techniques when possible but also introducing new elements when needed. We also consider previously developed models by some of the use case providers as starting point. Our proposed models are mainly supported by the provided domain/expert knowledge. But we also employ some data analysis tools to try to validate as much as possible this domain/expert knowledge, because the complexity of of the modelled systems/processes might cause that some of the elicited knowledge to no be completely accurate. We also try to address these situations. 

Finally, using the combination of the three use-case-tailored static and dynamic models, we define a preliminary AMIDST modelling framework, which is presented as a dynamic extension of a firstly introduced static framework. In this way, we claim that this framework is able to tackle all use-cases with different data-driven learning methods, as well as to take advantage of structural adjustments with respect to a particular domain.

It is crucial here to note that the current AMIDST modelling framework should be considered as an initial proposal. The proposal might be updated/extended in the future in the light of evaluation results of the proposed model as well as the impact of the study of the inference and learning techniques for dealing with massive data streams. The final AMIDST modelling framework will be presented in Deliverable 2.2.


