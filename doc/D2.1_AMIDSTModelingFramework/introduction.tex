\section{Introduction}

Data is being generated at a tremendous rate, e.g., with sensors providing continuous measurements of the environments in which they operate. To handle these extremely large data streams, current systems often employ simplistic solution techniques that, e.g., only consider the most recently generated data or only store the data at a much lower frequency than with which it is generated. By doing so, potentially valuable data, which could otherwise have contributed to an improved system accuracy, is being ignored. 

In this document, we analyse three particular domains facing extremely large datasets, namely risk prediction in credit operations, real time event detection in oil-well drilling and real time identification and interpretation of manoeuvres in traffic. They will serve as a guide to design the general AMIDST modelling framework to provide a scalable framework that facilitates efficient analysis and prediction based on streaming data. Even though we are looking at extremely large sets of streaming data, the model representation is "manageable" on a "desktop computer". The AMIDST modelling framework will be designed as a probabilistic graphical model, which entails several advantages: 1) uncertainty about predictions can be quantified; 2) missing and erroneous data records and observations can be handled using robust techniques based on well-founded principled methods; and 3) the model itself serves as a compact representation of the data, thus reducing the necessity of storing huge volumes of historical data (as the model is continuously updated with new data).

The structure of the document is as follows: Section \ref{Section:Preliminaries} introduces the required background and notation, Section \ref{Section:PreliminaryModels} describes the different domains and probabilistic graphical models proposed for each of them. Section \ref{section:AMIDSTmodelClass} presents and discusses the proposed general AMIDST model class. Finally, Section \ref{conclusions} includes a brief description of the ongoing and future work.