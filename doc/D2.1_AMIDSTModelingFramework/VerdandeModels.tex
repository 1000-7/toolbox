\subsection{Verdande Models}
\label{Section:VerdandeModels}


%\subsubsection{Introduction}

As has been pointed out in Delivrable D1.2, three main tasks have to be addressed for Verdande Technology use case:

\begin{enumerate}

\item \textbf{Automatic detection of drill string vibrations and abnormal torque states} which aims to better diagnose the shape of the wellbore and the state of the equipment, make better decisions on how to manage the well, and thereby reduce the non-productive time. To this end, a probabilistic graphical model for erratic torque monitoring and detection will be used.


\item \textbf{Semi-automatic labelling}: Given unlabelled data streams collected over time from typical drilling conditions, semi-automatic labelleing aims to compute a normality score for each considered drilling situation, then label it as either "normal" or "abnormal". As previous task, a probabilistic graphical model will be used, taking into account of the temporal dynamics of the drilling process and continuously adapting to changes in the incoming streaming data. Semi-automatic labelling allows to reduce the non-productive time and improve as well the data quality.

\item \textbf{Automatic formation detection}: which aims to predict in real time the formation tops from the MWD (measurements while drilling) data using a probabilistic graphical model. Once again, this should be performed taking into account of the temporal dynamics of the drilling process and continuously adapting to changes in the incoming streaming data. The automatic formation detection is vital for dealing with several issues such as hole instability and vibrations, and also important for reducing the costs and the overall non-productive time.
\end{enumerate}

For all tasks, the model must deal with both continuous and discrete observed random variables. Moreover, the oil-well data to be used presents the following characteristics: 

\begin{itemize}

\item It has a dynamic structure consisting of \emph{long-term} patches (ranging from a couple of hundred observations and into the thousands). Inside a patch, the data is typically fairly \emph{stable} and with low noise, even if this is not always the case. Between the patches the data can vary a lot. A \emph{patch of data} typically corresponds to the implementation of one activity (like drilling, connection tripping in/out, etc.). Consequently, the models would be better designed locally inside each single patch.

\item Inside one activity, many of the attributes can be strongly correlated. The observed correlation between variables $X$ and $Y$ can either be instantaneous (i.e., corr($X_t, Y_t$) significant), or delayed to some extent (i.e., exposed through the correlation corr($X_t, Y_{t+k}$) for some fixed $k$).

\item Physical models can be used to understand why these correlations are there, but not to quantify them. In addition, sometimes the strength, and even the sign, of the correlation may change from well to well.

\end{itemize}



\subsubsection{Detection of drill string vibrations}


\subsubsection{Semi-automatic labelling}


\subsubsection{Automatic formation detection}


\subsubsection{Discussion and future models}
