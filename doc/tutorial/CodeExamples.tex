% !TEX root = TexMain.tex
%----------------------------------------------------------------------------------------------------------------------------------------
\chapter{Code Examples of the Core API}\label{sec:codeExamples}
%----------------------------------------------------------------------------------------------------------------------------------------


\section{Data Streams}


In this example we show how to use the main features of a \textit{DataStream} object. More precisely,  we show six different ways of iterating over the data samples of a \textit{DataStream} object.



\begin{lstlisting}
//We can open the data stream using the static class DataStreamLoader
DataStream<DataInstance> data = DataStreamLoader.openFromFile("datasets/SmallDataSet.arff");

//Access to the attributes defining the data set
System.out.println("Attributes defining the data set");
for (Attribute attribute : data.getAttributes()) {
    System.out.println(attribute.getName());
}
Attribute attA = data.getAttributes().getAttributeByName("A");

//1. Iterating over samples using a for loop
System.out.println("1. Iterating over samples using a for loop");
for (DataInstance dataInstance : data) {
    System.out.println("The value of attribute A for the current data instance is: " + dataInstance.getValue(attA));
}


//2. Iterating using streams. We need to restart the data again as a DataStream can only be used once.
System.out.println("2. Iterating using streams.");
data.restart();
data.stream().forEach(dataInstance ->
                System.out.println("The value of attribute A for the current data instance is: " + dataInstance.getValue(attA))
);


//3. Iterating using parallel streams.
System.out.println("3. Iterating using parallel streams.");
data.restart();
data.parallelStream(10).forEach(dataInstance ->
                System.out.println("The value of attribute A for the current data instance is: " + dataInstance.getValue(attA))
);

//4. Iterating over a stream of data batches.
System.out.println("4. Iterating over a stream of data batches.");
data.restart();
data.streamOfBatches(10).forEach(batch -> {
    for (DataInstance dataInstance : batch)
        System.out.println("The value of attribute A for the current data instance is: " + dataInstance.getValue(attA));
});

//5. Iterating over a parallel stream of data batches.
System.out.println("5. Iterating over a parallel stream of data batches.");
data.restart();
data.parallelStreamOfBatches(10).forEach(batch -> {
    for (DataInstance dataInstance : batch)
        System.out.println("The value of attribute A for the current data instance is: " + dataInstance.getValue(attA));
});


//6. Iterating over data batches using a for loop
System.out.println("6. Iterating over data batches using a for loop.");
for (DataOnMemory<DataInstance> batch : data.iterableOverBatches(10)) {
    for (DataInstance dataInstance : batch)
        System.out.println("The value of attribute A for the current data instance is: " + dataInstance.getValue(attA));
}

\end{lstlisting}


\section{Variables}


This example show the basic functionality of the classes Variables and Variable.

\begin{lstlisting}
//We first create an empty Variables object
Variables variables = new Variables();

//We invoke the "new" methods of the object Variables to create new variables.
//Now we create a Gaussian variables
Variable gaussianVar = variables.newGaussianVariable("Gaussian");

//Now we create a Multinomial variable with two states
Variable multinomialVar = variables.newMultionomialVariable("Multinomial", 2);

//Now we create a Multinomial variable with two states: TRUE and FALSE
Variable multinomialVar2 = variables.newMultionomialVariable("Multinomial2", Arrays.asList("TRUE, FALSE"));

//For Multinomial variables we can iterate over their different states
FiniteStateSpace states = multinomialVar2.getStateSpaceType();
states.getStatesNames().forEach(System.out::println);

//Variable objects can also be used, for example, to know if one variable can be set as parent of some other variable
System.out.println("Can a Gaussian variable be parent of Multinomial variable? " +
        (multinomialVar.getDistributionType().isParentCompatible(gaussianVar)));

System.out.println("Can a Multinomial variable be parent of Gaussian variable? " +
        (gaussianVar.getDistributionType().isParentCompatible(multinomialVar)));
\end{lstlisting}



\section{Handling Bayesian Networks}


\subsection{Creating Bayesian Networks with no hidden variables}

In this example, we take a data set, create a BN and we compute the log-likelihood of all the samples
of this data set. The numbers defining the probability distributions of the BN are randomly fixed.

\begin{lstlisting}
//We can open the data stream using the static class DataStreamLoader
DataStream<DataInstance> data = DataStreamLoader.openFromFile("datasets/syntheticData.arff");

/**
 * 1. Once the data is loaded, we create a random variable for each of the attributes 
 * (i.e. data columns) in our data.
 *
 * 2. StaticVariables is the class for doing that. It takes a list of Attributes and 
 * internally creates all the variables. We create the variables using StaticVariables 
 * class to guarantee that each variable has a different ID number and make it 
 * transparent for the user.
 *
 * 3. We can extract the Variable objects by using the method 
 * getVariableByName();
 */
Variables variables = new Variables(data.getAttributes());

Variable a = variables.getVariableByName("A");
Variable b = variables.getVariableByName("B");
Variable c = variables.getVariableByName("C");
Variable d = variables.getVariableByName("D");
Variable e = variables.getVariableByName("E");
Variable g = variables.getVariableByName("G");
Variable h = variables.getVariableByName("H");
Variable i = variables.getVariableByName("I");

/**
 * 1. Once you have defined your StaticVariables object, the next step is to create
 * a DAG structure over this set of variables.
 *
 * 2. To add parents to each variable, we first recover the ParentSet object by the 
 * method getParentSet(Variable var) and then call the method addParent().
 */
DAG dag = new DAG(variables);

dag.getParentSet(e).addParent(a);
dag.getParentSet(e).addParent(b);

dag.getParentSet(h).addParent(a);
dag.getParentSet(h).addParent(b);

dag.getParentSet(i).addParent(a);
dag.getParentSet(i).addParent(b);
dag.getParentSet(i).addParent(c);
dag.getParentSet(i).addParent(d);

dag.getParentSet(g).addParent(c);
dag.getParentSet(g).addParent(d);

/**
 * 1. We first check if the graph contains cycles.
 *
 * 2. We print out the created DAG. We can check that everything is as expected.
 */
if (dag.containCycles()) {
    try {
    } catch (Exception ex) {
        throw new IllegalArgumentException(ex);
    }
}

System.out.println(dag.toString());


/**
 * 1. We now create the Bayesian network from the previous DAG.
 *
 * 2. The BN object is created from the DAG. It automatically looks at the 
 * distribution type of each variable and their parents to initialize the 
 * Distributions objects that are stored inside (i.e. Multinomial, Normal, 
 * CLG, etc). The parameters defining these distributions are properly 
 * initialized.
 *
 * 3. The network is printed and we can have look at the kind of 
 * distributions stored in the BN object.
 */
BayesianNetwork bn = BayesianNetwork.newBayesianNetwork(dag);
System.out.println(bn.toString());


/**
 * 1. We iterate over the data set sample by sample.
 *
 * 2. For each sample or DataInstance object, we compute the log of the probability 
 * that the BN object assigns to this observation.
 *
 * 3. We accumulate these log-probs and finally we print the log-prob of the data set.
 */
double logProb = 0;
for (DataInstance instance : data) {
    logProb += bn.getLogProbabiltyOf(instance);
}
System.out.println(logProb);

BayesianNetworkWriter.saveToFile(bn, "networks/huginStaticBNExample.bn");
\end{lstlisting}


\subsection{Creating Bayesian Networks with hidden variables}

In this example, we simply show how to create a BN model with hidden variables. We simply create a BN for clustering, i.e.,  a naive-Bayes like structure with a single common hidden variable acting as parant of all the observable variables.
 
\begin{lstlisting}

//We can open the data stream using the static class DataStreamLoader
DataStream<DataInstance> data = DataStreamLoader.openFromFile("datasets/syntheticData.arff");

/**
 * 1. Once the data is loaded, we create a random variable for each of the attributes 
 * (i.e. data columns) in our data.
 *
 * 2. StaticVariables is the class for doing that. It takes a list of Attributes and 
 * internally creates all the variables. We create the variables using StaticVariables 
 * class to guarantee that each variable has a different ID number and make it 
 * transparent for the user.
 *
 * 3. We can extract the Variable objects by using the method 
 * getVariableByName();
 */
Variables variables = new Variables(data.getAttributes());

Variable a = variables.getVariableByName("A");
Variable b = variables.getVariableByName("B");
Variable c = variables.getVariableByName("C");
Variable d = variables.getVariableByName("D");
Variable e = variables.getVariableByName("E");
Variable g = variables.getVariableByName("G");
Variable h = variables.getVariableByName("H");
Variable i = variables.getVariableByName("I");

/**
 * 1. We create the hidden variable. For doing that we make use of the class 
 * VariableBuilder. When a variable is created from an Attribute object, it 
 * contains all the information we need (e.g. the name, the type, etc). But 
 * hidden variables does not have an associated attribute and, for this 
 * reason, we use now this VariableBuilder to provide this information to
 * StaticVariables object.
 *
 * 2. Using VariableBuilder, we define a variable called HiddenVar, which is not 
 * observable (i.e. hidden), its state space is a finite set with two elements, 
 * and its distribution type is multinomial.
 *
 * 3. We finally create the hidden variable using the method "newVariable".
 */

Variable hidden = variables.newMultionomialVariable("HiddenVar", Arrays.asList("TRUE", "FALSE"));

/**
 * 1. Once we have defined your StaticVariables object, including the hidden 
 * variable, the next step is to create a DAG structure over this set of variables.
 *
 * 2. To add parents to each variable, we first recover the ParentSet object by 
 * the method getParentSet(Variable var) and then call the method 
 * addParent(Variable var).
 *
 * 3. We just put the hidden variable as parent of all the other variables. Following 
 * a naive-Bayes like structure.
 */
DAG dag = new DAG(variables);

dag.getParentSet(a).addParent(hidden);
dag.getParentSet(b).addParent(hidden);
dag.getParentSet(c).addParent(hidden);
dag.getParentSet(d).addParent(hidden);
dag.getParentSet(e).addParent(hidden);
dag.getParentSet(g).addParent(hidden);
dag.getParentSet(h).addParent(hidden);
dag.getParentSet(i).addParent(hidden);

/**
 * We print the graph to see if is properly created.
 */
System.out.println(dag.toString());

/**
 * 1. We now create the Bayesian network from the previous DAG.
 *
 * 2. The BN object is created from the DAG. It automatically looks at the 
 * distribution type of each variable and their parents to initialize the 
 * Distributions objects that are stored inside (i.e. Multinomial, Normal, 
 * CLG, etc). The parameters defining these distributions are properly 
 * initialized.
 *
 * 3. The network is printed and we can have look at the kind of 
 * distributions stored in the BN object.
 */
BayesianNetwork bn = BayesianNetwork.newBayesianNetwork(dag);
System.out.println(bn.toString());

/**
 * Finally the Bayesian network is saved to a file.
 */
BayesianNetworkWriter.saveToFile(bn, "networks/huginStaticBNHiddenExample.bn");

\end{lstlisting}

\subsection{Modifying Bayesian Networks}

In this example we show how to access and modify the conditional probabilities of a Bayesian network model.

\begin{lstlisting}
//We first generate a Bayesian network with one multinomial, one Gaussian variable and one link
BayesianNetworkGenerator.setNumberOfGaussianVars(1);
BayesianNetworkGenerator.setNumberOfMultinomialVars(1,2);
BayesianNetworkGenerator.setNumberOfLinks(1);

BayesianNetwork bn = BayesianNetworkGenerator.generateBayesianNetwork();

//We print the randomly generated Bayesian networks
System.out.println(bn.toString());

//We first access the variable we are interested in
Variable multiVar = bn.getStaticVariables().getVariableByName("DiscreteVar0");

//Using the above variable we can get the associated distribution and modify it
Multinomial multinomial = bn.getConditionalDistribution(multiVar);
multinomial.setProbabilities(new double[]{0.2, 0.8});

//Same than before but accessing the another variable
Variable normalVar = bn.getStaticVariables().getVariableByName("GaussianVar0");

//In this case, the conditional distribtuion is of the type "Normal given Multinomial Parents"
Normal_MultinomialParents normalMultiDist = bn.getConditionalDistribution(normalVar);
normalMultiDist.getNormal(0).setMean(1.0);
normalMultiDist.getNormal(0).setVariance(1.0);

normalMultiDist.getNormal(1).setMean(0.0);
normalMultiDist.getNormal(1).setVariance(1.0);

//We print modified Bayesian network
System.out.println(bn.toString());

\end{lstlisting}


\section{I/O Functionality}

\subsection{I/O of Data Streams}

In this example we show how to load and save data sets from ".arff" files (http://www.cs.waikato.ac.nz/ml/weka/arff.html)

\begin{lstlisting}
//We can open the data stream using the static class DataStreamLoader
DataStream<DataInstance> data = DataStreamLoader.openFromFile("datasets/syntheticData.arff");

//We can save this data set to a new file using the static class DataStreamWriter
DataStreamWriter.writeDataToFile(data, "datasets/tmp.arff");
\end{lstlisting}

\subsection{I/O of Bayesian Networks}

In this example we show how to load and save Bayesian networks models for a binary file with ".bn" extension. In this toolbox Bayesian networks models are saved as serialized objects.

\begin{lstlisting}
//We can load a Bayesian network using the static class BayesianNetworkLoader
BayesianNetwork bn = BayesianNetworkLoader.loadFromFile("./networks/WasteIncinerator.bn");

//Now we print the loaded model
System.out.println(bn.toString());

//Now we change the parameters of the model
bn.randomInitialization(new Random(0));

//We can save this Bayesian network to using the static class BayesianNetworkWriter
BayesianNetworkWriter.saveToFile(bn, "networks/tmp.bn");
\end{lstlisting}


\section{Inference Algorithms}

\subsection{The Inference Engine}
This example show how to perform inference in a Bayesian network model using the InferenceEngine static class. This class aims to be a straigthfoward way to perform queries over a Bayesian network model. By the default the \textit{VMP} inference method is invoked.

\begin{lstlisting}
//We first load the WasteIncinerator bayesian network which has multinomial 
//and Gaussian variables.
BayesianNetwork bn = BayesianNetworkLoader.loadFromFile("./networks/WasteIncinerator.bn");

//We recover the relevant variables for this example: Mout which is normally 
//distributed, and W which is multinomial.
Variable varMout = bn.getStaticVariables().getVariableByName("Mout");
Variable varW = bn.getStaticVariables().getVariableByName("W");

//Set the evidence.
Assignment assignment = new HashMapAssignment(1);
assignment.setValue(varW,0);

//Then we query the posterior of
System.out.println("P(Mout|W=0) = " + InferenceEngine.getPosterior(varMout, bn, assignment));

//Or some more refined queries
System.out.println("P(0.7<Mout<6.59 | W=0) = " + InferenceEngine.getExpectedValue(varMout, bn, v -> (0.7 < v && v < 6.59) ? 1.0 : 0.0 ));
\end{lstlisting}

\subsection{Variational Message Passing}

This example we show how to perform inference on a general Bayesian network using the Variational Message Passing (VMP)
algorithm detailed in
\vspace{0.2cm}
\textit{Winn, J. M., Bishop, C. M. (2005). Variational message passing. In Journal of Machine Learning Research (pp. 661-694).
}


\begin{lstlisting}
//We first load the WasteIncinerator bayesian network which has multinomial 
//and Gaussian variables.
BayesianNetwork bn = BayesianNetworkLoader.loadFromFile("./networks/WasteIncinerator.bn");

//We recover the relevant variables for this example: Mout which is normally 
//distributed, and W which is multinomial.
Variable varMout = bn.getStaticVariables().getVariableByName("Mout");
Variable varW = bn.getStaticVariables().getVariableByName("W");

//First we create an instance of a inference algorithm. In this case, we use 
//the VMP class.
InferenceAlgorithm inferenceAlgorithm = new VMP();

//Then, we set the BN model
inferenceAlgorithm.setModel(bn);

//If exists, we also set the evidence.
Assignment assignment = new HashMapAssignment(1);
assignment.setValue(varW,0);
inferenceAlgorithm.setEvidence(assignment);

//Then we run inference
inferenceAlgorithm.runInference();

//Then we query the posterior of
System.out.println("P(Mout|W=0) = " + inferenceAlgorithm.getPosterior(varMout));

//Or some more refined queries
System.out.println("P(0.7<Mout<6.59 | W=0) = " + inferenceAlgorithm.getExpectedValue(varMout, v -> (0.7 < v && v < 6.59) ? 1.0 : 0.0 ));

//We can also compute the probability of the evidence
System.out.println("P(W=0) = "+Math.exp(inferenceAlgorithm.getLogProbabilityOfEvidence()));
\end{lstlisting}

\subsection{Importance Sampling}

This example we show how to perform inference on a general Bayesian network using an importance sampling
algorithm detailed in

\vspace{0.2cm}
\textit{Fung, R., Chang, K. C. (2013). Weighing and integrating evidence for stochastic simulation in Bayesian networks. arXiv preprint arXiv:1304.1504.
}

\begin{lstlisting}
//We first load the WasteIncinerator bayesian network which has multinomial 
//and Gaussian variables.
BayesianNetwork bn = BayesianNetworkLoader.loadFromFile("./networks/WasteIncinerator.bn");

//We recover the relevant variables for this example: Mout which is normally 
//distributed, and W which is multinomial.
Variable varMout = bn.getStaticVariables().getVariableByName("Mout");
Variable varW = bn.getStaticVariables().getVariableByName("W");

//First we create an instance of a inference algorithm. In this case, we use 
//the ImportanceSampling class.
InferenceAlgorithm inferenceAlgorithm = new ImportanceSampling();

//Then, we set the BN model
inferenceAlgorithm.setModel(bn);

//If exists, we also set the evidence.
Assignment assignment = new HashMapAssignment(1);
assignment.setValue(varW,0);
inferenceAlgorithm.setEvidence(assignment);

//We can also set to be run in parallel on multicore CPUs
inferenceAlgorithm.setParallelMode(true);

//Then we run inference
inferenceAlgorithm.runInference();

//Then we query the posterior of
System.out.println("P(Mout|W=0) = " + inferenceAlgorithm.getPosterior(varMout));

//Or some more refined queries
System.out.println("P(0.7<Mout<6.59 | W=0) = " + inferenceAlgorithm.getExpectedValue(varMout, v -> (0.7 < v && v < 6.59) ? 1.0 : 0.0 ));

//We can also compute the probability of the evidence
System.out.println("P(W=0) = "+Math.exp(inferenceAlgorithm.getLogProbabilityOfEvidence()));
\end{lstlisting}


\section{Learning Algorithms}

\subsection{Parallel Maximum Likelihood}

This example shows how to learn in parallel the parameters of a Bayesian network from a stream of data using maximum likelihood.
\begin{lstlisting}
//We can open the data stream using the static class DataStreamLoader
DataStream<DataInstance> data = DataStreamLoader.openFromFile("datasets/syntheticData.arff");

//We create a MaximumLikelihood object with the MaximumLikehood builder
MaximumLikelihood parameterLearningAlgorithm = new MaximumLikelihood();

//We activate the parallel mode.
parameterLearningAlgorithm.setParallelMode(true);

//We fix the DAG structure
parameterLearningAlgorithm.setDAG(getNaiveBayesStructure(data,0));

//We set the batch size which will be employed to learn the model in parallel
parameterLearningAlgorithm.setBatchSize(100);

//We set the data which is going to be used for leaning the parameters
parameterLearningAlgorithm.setDataStream(data);


//We perform the learning
parameterLearningAlgorithm.runLearning();

//And we get the model
BayesianNetwork bnModel = parameterLearningAlgorithm.getLearntBayesianNetwork();

//We print the model
System.out.println(bnModel.toString());
\end{lstlisting}

\subsection{Streaming Variational Bayes}

This examples shows how to learn in the parameters of a Bayesian network from a stream of data with a Bayesian
approach using the following algorithm

\textit{Broderick, T., Boyd, N., Wibisono, A., Wilson, A. C., \& Jordan, M. I. (2013). Streaming variational Bayes. 
In Advances in Neural Information Processing Systems (pp. 1727-1735).
}



\subsubsection*{Version 1}

In this first example we show a simple way to invoke this learning algorithm,


\begin{lstlisting}
//We can open the data stream using the static class DataStreamLoader
DataStream<DataInstance> data = DataStreamLoader.openFromFile("datasets/syntheticData.arff");

//We create a StreamingVariationalBayesVMP object
StreamingVariationalBayesVMP parameterLearningAlgorithm = new StreamingVariationalBayesVMP();

//We fix the DAG structure
parameterLearningAlgorithm.setDAG(getHiddenNaiveBayesStructure(data));

//We fix the size of the window, which must be equal to the size of the data batches we use for learning
parameterLearningAlgorithm.setWindowsSize(5);

//We set the data which is going to be used for leaning the parameters
parameterLearningAlgorithm.setDataStream(data);

//We perform the learning
parameterLearningAlgorithm.runLearning();

//And we get the model
BayesianNetwork bnModel = parameterLearningAlgorithm.getLearntBayesianNetwork();

//We print the model
System.out.println(bnModel.toString());
\end{lstlisting}


\subsubsection*{Version 2}

In this second example we show a alternative use of this class which explicitly update the model by batches, 


\begin{lstlisting}
//We can open the data stream using the static class DataStreamLoader
DataStream<DataInstance> data = DataStreamLoader.openFromFile("datasets/syntheticData.arff");

//We create a StreamingVariationalBayesVMP object
StreamingVariationalBayesVMP parameterLearningAlgorithm = new StreamingVariationalBayesVMP();

//We fix the DAG structure
parameterLearningAlgorithm.setDAG(getHiddenNaiveBayesStructure(data));

//We fix the size of the window, which must be equal to the size of the data batches 
//we use for learning
parameterLearningAlgorithm.setWindowsSize(5);


//We should invoke this method before processing any data
parameterLearningAlgorithm.initLearning();


//Then we show how we can perform parameter learnig by a sequential updating of 
//data batches.
for (DataOnMemory<DataInstance> batch : data.iterableOverBatches(5)){
    parameterLearningAlgorithm.updateModel(batch);
}

//And we get the model
BayesianNetwork bnModel = parameterLearningAlgorithm.getLearntBayesianNetwork();

//We print the model
System.out.println(bnModel.toString());
\end{lstlisting}


% !TEX root = TexMain.tex
%----------------------------------------------------------------------------------------------------------------------------------------
\chapter{Code Examples of the HuginLink API}\label{sec:codeExamples}
%----------------------------------------------------------------------------------------------------------------------------------------

\section{Models conversion between AMIDST and Hugin}

This example shows how to use the class BNConverterToAMIDST and BNConverterToHugin to convert a 
Bayesian network models between Hugin and AMIDST formats


\begin{lstlisting}
//We load from Hugin format
Domain huginBN = BNLoaderFromHugin.loadFromFile("networks/asia.net");

//Then, it is converted to AMIDST BayesianNetwork object
BayesianNetwork amidstBN = BNConverterToAMIDST.convertToAmidst(huginBN);

//Then, it is converted to Hugin Bayesian Network object
huginBN = BNConverterToHugin.convertToHugin(amidstBN);

System.out.println(amidstBN.toString());
System.out.println(huginBN.toString());
\end{lstlisting}



\section{I/O of Bayesian Networks}

This example shows how to use the class BNLoaderFromHugin and BNWriterToHugin classes to load and
write Bayesian networks in Hugin format.

\begin{lstlisting}
//We load from Hugin format
Domain huginBN = BNLoaderFromHugin.loadFromFile("networks/asia.net");

//We save a AMIDST BN to Hugin format
BayesianNetwork amidstBN = BNConverterToAMIDST.convertToAmidst(huginBN);
BNWriterToHugin.saveToHuginFile(amidstBN,"networks/tmp.net");
\end{lstlisting}

\section{Invoking Hugin's inference engine}

This example we show how to perform inference using Hugin (http://www.hugin.com) inference engine within the AMIDST toolbox

\begin{lstlisting}
//We first load the WasteIncinerator bayesian network which has multinomial and Gaussian variables.
BayesianNetwork bn = BayesianNetworkLoader.loadFromFile("./networks/WasteIncinerator.bn");

//We recover the relevant variables for this example: Mout which is normally distributed, and W which is multinomial.
Variable varMout = bn.getStaticVariables().getVariableByName("Mout");
Variable varW = bn.getStaticVariables().getVariableByName("W");

//First we create an instance of a inference algorithm. In this case, we use the ImportanceSampling class.
InferenceAlgorithm inferenceAlgorithm = new HuginInference();
//Then, we set the BN model
inferenceAlgorithm.setModel(bn);

//If exists, we also set the evidence.
Assignment assignment = new HashMapAssignment(1);
assignment.setValue(varW,0);
inferenceAlgorithm.setEvidence(assignment);

//Then we run inference
inferenceAlgorithm.runInference();

//Then we query the posterior of
System.out.println("P(Mout|W=0) = " + inferenceAlgorithm.getPosterior(varMout));

//Or some more refined queries
System.out.println("P(0.7<Mout<3.5 | W=0) = " + inferenceAlgorithm.getExpectedValue(varMout, v -> (0.7 < v && v < 3.5) ? 1.0 : 0.0 ));

\end{lstlisting}

%%% Local Variables:
%%% mode: latex
%%% TeX-master: "TexMain"
%%% End:

\begin{lstlisting}


\end{lstlisting}