%\theoremstyle{definition}
%\newtheorem{definition}{Definition}

%\theoremstyle{theorem}
%\newtheorem{theorem}{Theorem}
%\theoremstyle{example}
%\newtheorem{example}{Example}
%\newtheorem{proposition}{Proposition}

\definecolor{verdoso}{rgb}{0,0.7,0}
\newcommand{\fondoverde}[1]{\multicolumn{1}{>{\columncolor{white}}c}{#1}}
\newcommand{\fondoOliveGreen}[1]{\multicolumn{1}{>{\columncolor{OliveGreen}}c}{#1}}
\newcommand{\fondorojo}[1]{\multicolumn{1}{>{\columncolor{red}}c}{#1}}
\newcommand{\amari}[1]{\multicolumn{1}{>{\columncolor{nodecolor}}r}{#1}}
\newcommand{\w}{\mathbf{w}}
\newcommand{\x}{\mathbf{x}}
\newcommand{\y}{\mathbf{y}}
\newcommand{\z}{\mathbf{z}}
\newcommand{\e}{\mathbf{e}}
\newcommand{\be}{\mathbf{b}}


\newcommand{\A}{\mathbf{A}}
\newcommand{\W}{\mathbf{W}}
\newcommand{\X}{\mathbf{X}}
\newcommand{\Y}{\mathbf{Y}}
\newcommand{\Z}{\mathbf{Z}}
\newcommand{\E}{\mathbf{E}}



\newcommand{\Reals}{I\!\! R}
\newcommand{\ele}{\mathbf{l}}
\newcommand{\nai}{na\"{\i}ve~}
\newcommand{\Nai}{Na\"{\i}ve~}
\newcommand{\peq}{\scriptsize}
\newcommand{\de}{\mathbf{d}}
\newcommand{\Var}{\mathrm{Var}}
\newcommand{\Cov}{\mathrm{Cov}}
\newcommand{\dom}{\mathrm{dom}}
\newcommand{\T}{{\cal T}}
\newcommand{\trans}{^{\mathsf{T}}}
\providecommand{\abs}[1]{\lvert#1\rvert}


\newtheorem{proposition}{Proposition}%[section]
%%%%%% MIS COMANDOS GRAFICOS %%%%%%%%%%%%%%%%%%




%\newgray{lightgris}{0.8}
%\newgray{migris}{0.7}
\newcommand{\nborde}[4]{\rput(#1,#2){\rnode{#4}{{\makebox{\centering \scriptsize #3}}}}}
 
 \newgray{grisclaro}{0.9}
  \newgray{grismuyclaro}{0.95}
 
\definecolor{nodecolor}{RGB}{238,221,130}

\definecolor{orange}{RGB}{255,127,0}


\newcommand{\rnod}[4]{\cnode(#1,#2){#3}{N#1-#2}\rput(#1,#2){\tiny #4}}
%\rnod{coorx}{coory}{rad}{text} % :Crea un nodo (con nombre "Ncoorx-coory") en (coorx,coory) de radio rad
\newcommand{\nod}[3]{\cnode(#1,#2){1}{N#1-#2}\rput(#1,#2){\tiny #3}}
%\nod{coorx}{coory}{text}   % :Crea un nodo (con nombre "Ncoorx-coory") en (coorx,coory) de radio la unidad de longitud
\newcommand{\nnod}[4]{\cnode(#1,#2){1.2}{#4}\rput(#1,#2){\footnotesize #3}}
\newcommand{\nnodtwolines}[4]{\cnode[doubleline=true](#1,#2){1.4}{#4}\rput(#1,#2){\footnotesize #3}}

\newcommand{\nnodgris}[4]{\cnode[fillstyle=solid, fillcolor=grisclaro](#1,#2){1.2}{#4}\rput(#1,#2){\footnotesize #3}}

%\nod{coorx}{coory}{text}   % :Crea un nodo (con nombre "Ncoorx-coory") en (coorx,coory) de radio la unidad de longitud con relleno grisclaro



%\nod{coorx}{coory}{text}   % :Crea un nodo (con nombre "Ncoorx-coory") en (coorx,coory) de radio la unidad de longitud y con su linea con estilo dashed
\newcommand{\nnoddashed}[4]{\cnode[linestyle=dashed, fillcolor=white](#1,#2){1.2}{#4}\rput(#1,#2){%
\footnotesize #3}}

 %\nnod{coorx}{coory}{text}{nom}   % :Crea un nodo (con nombre nom) en (coorx,coory) de radio la unidad de longitud
\newcommand{\anod}[3]{\rput(#1,#2){\circlenode{N#1-#2}{%
\tiny #3}}}
%\anod{coorx}{coory}{tex} %  :Crea un nodo (con nombre "Ncoorx-coory") en (coorx,coory) de radio ajustable al texto

\newcommand{\fle}[4]{\ncline[linecolor=black,linewidth=1pt,angleA=#2,angleB=#4, arrowsize = 3pt 3]{->}{#1}{#3}}


\newcommand{\fledashed}[4]{\ncline[linestyle=dashed,linecolor=black,linewidth=1pt,angleA=#2,angleB=#4, , arrowsize = 3pt 3]{->%
}{#1}{#3}}



%\fle{nod1}{ang1}{nod2}{ang2} % Dibuja una flecha desde el nod1 con salida en \'{a}ngulo ang1 al nod2 con entrada en \'{a}ngulo ang2
\newcommand{\ari}[4]{\ncline[linecolor=black,linewidth=1pt,angleA=#2,angleB=#4]{-%
}{#1}{#3}}
%\ari{nod1}{ang1}{nod2}{ang2} % Dibuja una l\'{\i}nea (arista) desde el nod1 con salida en \'{a}ngulo ang1 al nod2 con entrada en \'{a}ngulo ang2
\newcommand{\fledas}[4]{\ncline[linecolor=black,linestyle=dashed,linewidth=1pt,angleA=#2,angleB=#4]{->%
}{#1}{#3}}
%\fledas{nod1}{ang1}{nod2}{ang2} % Dibuja una flecha dashed desde el nod1 con salida en \'{a}ngulo ang1 al nod2 con entrada en \'{a}ngulo ang2
\newcommand{\aridas}[4]{\ncline[linecolor=black,linestyle=dashed,linewidth=1pt,angleA=#2,angleB=#4]{-%
}{#1}{#3}}
%\aridas{nod1}{ang1}{nod2}{ang2} % Dibuja una l\'{\i}nea dashed desde el nod1 con salida en \'{a}ngulo ang1 al nod2 con entrada en \'{a}ngulo ang2
\newcommand{\aridot}[4]{\ncline[linecolor=black,linestyle=dotted,linewidth=1pt,angleA=#2,angleB=#4]{-%
}{#1}{#3}}
%\aridot{nod1}{ang1}{nod2}{ang2} % Dibuja una l\'{\i}nea de puntos desde el nod1 con salida en \'{a}ngulo ang1 al nod2 con entrada en \'{a}ngulo ang2

\newcommand{\aridasgris}[4]{\ncline[linecolor=grisclaro,linestyle=dashed,linewidth=1pt,angleA=#2,angleB=#4]{-%
}{#1}{#3}}
%\aridotgris{nod1}{ang1}{nod2}{ang2} % Dibuja una l\'{\i}nea discont\'{\i}nua color gris desde el nod1 con salida en \'{a}ngulo ang1 al nod2 con entrada en \'{a}ngulo ang2

\newcommand{\novalnod}[4]{\rput(#1,#2){\ovalnode{#4}{\footnotesize #3}}}
\newcommand{\novalnodtwolines}[4]{\rput(#1,#2){\ovalnode[doubleline=true]{#4}{\footnotesize #3}}}
 %\novalnod{coorx}{coory}{text}{nom}   % :Crea un nodo oval (con nombre nom) en (coorx,coory)

 %\novalnod{coorx}{coory}{text}{nom}{fillstyle}{fillcolor}   
 % :Crea un nodo oval (con nombre nom) en (coorx,coory) y con (fillstyle =solid,...) y (fillcolor)
 

\newcommand{\nfram}[6]{\rput(#1,#2){\rnode{#6}{\psframebox{\parbox[c][#3][c]{#4}{%
 \centering \tiny #5}}}}}
 %\nfram{coorx}{coory}{text}{nom}   % :Crea un nodo-caja (con nombre nom) en (coorx,coory) de
 %altura alt y longitud lon y dentro centra text

\newcommand{\nafram}[4]{\rput(#1,#2){\rnode{#4}{\psframebox[fillcolor=white]{\makebox{\centering \scriptsize #3}}}}}
 %\nafram{coorx}{coory}{text}{nom}   % :Crea un nodo-caja (con nombre nom) en (coorx,coory) de
 %y  dentro centra text

\newcommand{\aritexta}[5]{\ncline[linecolor=black,linewidth=1pt,angleA=#2,angleB=#4]{-%
}{#1}{#3}\naput{\tiny{#5}}}


\newcommand{\aritextb}[5]{\ncline[linecolor=black,linewidth=1pt,angleA=#2,angleB=#4]{-%
}{#1}{#3}\nbput{\tiny{#5}}}


%\newcommand{\novalnodtwolines}[4]{\rput(#1,#2){\ovalnode[doubleline=true]{#4}{\footnotesize #3}}}

%\newcommand{\novalnodtwolines}[4]{\cnode[doubleline=true](#1,#2){1.2}{#4}\rput(#1,#2){\footnotesize #3}}
%\cput[doubleline=true](1,.5){\large $K_1$}



%%%%%%%%%%%%%
%\newcommand{\idr}{\perp\!\!\!\perp}
%\renewcommand{\algorithmcfname}{Algoritmo}


%\theoremstyle{definition}
%\newtheorem{definicion}{Definición}[chapter]
%
%\theoremstyle{example}
%\newtheorem{ejemplo}{Ejemplo}[chapter]
%
%\theoremstyle{theorem}
%\newtheorem{teorema}{Teorema}[chapter]
%
%
%\theoremstyle{proposition}
%\newtheorem{proposicion}{Proposición}[chapter]
%

%\newcommand{\nnod}[4]{\cnode(#1,#2){1}{#4}\rput(#1,#2){\tiny #3}}


%\nod{coorx}{coory}{text}   % :Crea un nodo (con nombre "Ncoorx-coory") en (coorx,coory) de radio la unidad de longitud con relleno grisclaro


