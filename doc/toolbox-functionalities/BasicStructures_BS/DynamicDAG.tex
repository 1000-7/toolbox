\newpage
\subsection{Dynamic DAG}
\label{DynamicDAG:ID}

\begin{description}
\item[Deadline:] M12
\item[Responsible:] Hanen
\item[Code-Package:] \texttt{eu.amidst.core.models}
\end{description}

%--------------------------------------------------------------------------------------------
\subsubsection*{Description}
%--------------------------------------------------------------------------------------------

It defines the dynamic Bayesian network graphical structure over a list of dynamic variables by specifying their parent sets at time 0 and time T.

%--------------------------------------------------------------------------------------------
\subsubsection*{Detailed functionality}
%--------------------------------------------------------------------------------------------

\begin{itemize}
\item It defines the parent set at time 0 and time T for each dynamic variable.
\item It test and detect if a dynamic DAG contains cycles or not.
\end{itemize}

%--------------------------------------------------------------------------------------------
\subsubsection*{Code example}
%--------------------------------------------------------------------------------------------

\begin{table}[H]
\begin{tabular}{l} \hline

        \texttt{DynamicDAG dynamicDAG = new DynamicDAG(dynamicVariables);}\\\\
        
        \texttt{dynamicDAG.getParentSetTimeT(observedTRQ).addParent(observedWOB);}\\
        \texttt{dynamicDAG.getParentSetTimeT(observedTRQ).addParent(observedRPMB);}\\
        \texttt{dynamicDAG.getParentSetTimeT(observedTRQ).addParent(observedMFI);}\\
        \texttt{dynamicDAG.getParentSetTimeT(observedTRQ).addParent(realTRQ);}\\
        \texttt{dynamicDAG.getParentSetTimeT(observedTRQ).addParent(hidden);}\\
        \texttt{dynamicDAG.getParentSetTimeT(observedTRQ).addParent(mixture);}\\  \hline 

\end{tabular}
\end{table}




        
        
       
        
        
        