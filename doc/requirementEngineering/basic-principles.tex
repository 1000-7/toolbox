\section{Basic principles in requirement engineering}
\label{sec:stateOfArt}

In practice, the RE process ends up with a document containing a list of requirements, which are in the form of what a software must do or comply with.  

To date there is no common definition of RE.  Some definitions focus on elicitation of requirements and therefore the interaction with the user, while others focus on the documentation or the specification.  A definition that takes both focuses into account is the IEEE standard given in \cite{Iee90}:

\emph{
\begin{enumerate}
\item The process of studying user needs to arrive at a definition of system, hardware or software requirements.
\item The process of studying and refining system, hardware or software requirements.
\end{enumerate}
}

In the context of understanding the RE process, it is worth spending some space on defining a requirement itself.  A definition of a requirement is given in IEEE standard \cite{Iee90}: 
\emph{
\begin{enumerate}
\item A condition or capability needed by a user to solve a problem or achieve an objective. 
\item A condition or capability that must be met or possessed by a system or system component to satisfy a contract, standard, specification or other formally imposed document. 
\item A documented representation of a condition or capability as in 1 or 2.
\end{enumerate}
}

This definition has a clear focus on the user, the system/system component and also which contract, standard or specification is needed to be met. Notice, that the requirement is related to \emph{what} a system can do and not \emph{how} it is done.

\subsection{Activities involved in requirement engineering}

The activities involved in RE vary widely, depending on the type of system being developed and the specific practices of the organization(s) involved  \cite{Som11}.  These may include:
\begin{itemize}
\item Requirements elicitation 
\item Requirements analysis and negotiation - checking requirements and resolving stakeholders conflicts
\item Requirements specification - documenting the requirements in a requirements document
\item Requirements validation - checking that the documented requirements are consistent and meet stakeholders needs
\item Requirements management - managing changes to the requirements as the system is developed and put into use
\end{itemize}

These activities are sometimes presented as chronological stages although, in practice, there  are considerable interleaving between them.  

Anyone who has a direct or indirect influence on the process is identified as a stakeholder.   Stakeholders include end-users that will interact with the system, the developers that will maintain the system, management, domain experts, union representatives etcetera.  A challenge in the RE process is therefore to keep a smooth communication between the different stakeholders.  The use-case driven approach to RE focuses on simplyfing the communication between the end-users and the developers to improve the overall comminication.

\subsection{Use-case driven requirement engineering}
\label{sec:use-case-driven}
It has always been a challenge for the software industry to communicate functionality to the users of a software. Moreover, software engineers are often frustrated, because users often do not know what they want. They only have an idea of what they want.  To improve this communication, the use-case driven approach was developed in the nineties.  It was first published by Ivar Jacobsen \cite{Jac92} and more modern references are \cite{Poh10} and \cite{Coc01}.  A use-case focuses only on the interaction between a user and the system.  Requirements are always associated with a use-case. This means that the user is requested to only focus on what he/she wants.  This is an advantage, compared to the traditional way where requirements are listed in relation to components and subcomponents in the software.  The traditional way often lead to a complexity that the user do not understand.  Also, it is more common with requirement duplicates in the traditional approach.

A use-case is a list of steps, typically defining interactions between an actor and a system, to achieve a goal. The actor can be a human or an external system.  An overview on how to write effective use-cases is given in \cite{Coc01}, where several templates are given. The use-case providers are asked to provide the use-cases in natural language and for each use-case the following questions are central:

\begin{enumerate}
\item Who are the actors involved in the use-case? An actor is either a person or an entity that interacts with the software.  
\item What is the main event that initiates the use-case? This could e.g. be an external business event or a system event that causes the use-case to begin.  It could also be the initial step in a normal work flow. 
\item What are the main user actions and system responses that will take place during the normal execution of the use-case?. This dialog sequence will ultimately lead to accomplishing the goal that is implied by the use-case name and description.
\item How can we evaluate the success of the use-case?
\end{enumerate}
 
It is also common to group the users, or human actors, within an organization into a small set of user groups. The users
within each user group need to have similar roles within their organization and their set of competences are expected to
be similar. \cite{Could we expand a bit on this?} 

To understand the use-case driven approach better, it is useful to distinguish between functional and non functional
requirements.  Functional requirements are those requirements that are directly related to the interaction between the
user and the system.  The non functional requirements are more hidden for the user are related to the global overall
\fixme{How does the discussion about non functional requirements relate to our approach? Could we expand on this?}
success.  For instance scalability, traceability and testability.  When use-cases are provided and functional
requirements are identified, it is the requirement engineers role to identify, document and communicate these non
functional requirements as well.  The use-case driven approach to requirement engineering focuses on revealing the
functional requirements together with the users.  This improves the communication between the users and the developers,
because the focus is on what the users wants and less on how it can be done. 
