\section{Basic principles in requirements engineering}
\label{sec:stateOfArt}

The  RE process typically ends up with a document containing a list of
requirements for the system to be developed. This could, for instance, include what a software component must do or
comply with. To date there is no common and agreed-upon definition of RE. Some definitions focus on the elicitation of requirements and therefore the
interaction with the user, while others focus on the documentation or the specification.  A definition that takes both
foci into account is the IEEE standard \cite{Ieee90}, which defines requirements analysis as\emph{
\begin{enumerate}
\item The process of studying user needs to arrive at a definition of system, hardware or software requirements.
\item The process of studying and refining system, hardware or software requirements.
\end{enumerate}
}
In the context of understanding the RE process, a possible definition of a requirement is given in the IEEE standard \cite{Ieee90}: 
\emph{
\begin{enumerate}
\item A condition or capability needed by a user to solve a problem or achieve an objective. 
\item A condition or capability that must be met or possessed by a system or system component to satisfy a contract, standard, specification or other formally imposed document. 
\item A documented representation of a condition or capability as in 1 or 2.
\end{enumerate}
}
This definition has a clear focus on the user, the system/system component and also which contract, standard or
specification is needed to be met. Notice, that the requirement is related to \emph{what} a system can do and not
\emph{how} it is done. We will adopt the same perspective in the present document.

\subsection{Activities involved in requirements engineering}

The activities involved in RE vary widely, depending on the type of system being developed and the specific practices of the organization(s) involved  \cite{Som11}.  These may include:
\begin{itemize}
\item Requirements elicitation 
\item Requirements analysis and negotiation; checking requirements and resolving stakeholders conflicts
\item Requirements specification; documenting the requirements in a requirements document
\item Requirements validation; checking that the documented requirements are consistent and meet stakeholders needs
\item Requirements management; managing changes to the requirements as the system is developed and put into use
\end{itemize}

These activities are sometimes presented as chronological stages although, in practice, there  are considerable interleaving between them.  

Anyone who has a direct or indirect influence on the process or the system to be developed is identified as a stakeholder.   Stakeholders include end-users that will interact with the system, the developers that will maintain the system, management, domain experts, union representatives, etc..  A challenge in the RE process is therefore to keep a smooth communication between the different stakeholders.  The use-case driven approach to RE focuses on simplifying the communication between the end-users and the developers to improve the overall communication.

\subsection{Use-case driven requirements engineering}
\label{sec:use-case-driven}
It can often be a challenge for the future users of a software system to accurately communicate expectations about functionality to the
software developers. Moreover, this communication can be further hampered by the users not
being able to accurately express what functionality they want. To improve on this communication, a use-case driven
approach to RE was developed in the
nineties  \cite{Jac92,Poh10,Coc01}.  A use-case focuses only on the interaction between a user and the system to be
developed, and forms the basis for the requirements engineering where each requirement is
associated with a use-case. This means that when specifying the use cases and requirements, the user is requested to only focus on what he/she wants.  This is an
advantage, compared to the traditional RE approach, where requirements are listed in relation to
components or sub-components in
the software; an approach that often entails a degree of complexity that the user finds difficult to understand. \fixme{Do
  we have a reference for this?} 

A use-case consist of a list of steps, typically defining interactions between an actor and a system component, with the aim of
achieving a specific  goal; the actor can be either a human or an external system.  An overview on how to write
effective use-cases is given in \cite{Coc01}, where several templates are presented. Common for these templates is that
the users are asked to describe the use-cases in natural language and addressing the following questions:
\begin{enumerate}
\item Who are the actors involved in the use-case? An actor is either a person or an entity that interacts with the software.  
\item What is the main event that initiates the use-case? This could, e.g., be an external business event or a system event that causes the use-case to begin.  It could also be the initial step in a normal work flow. 
\item What are the main user actions and system responses that will take place during the normal execution of the use-case?. 
\item How can we evaluate the success of the use-case?
\end{enumerate}
This dialog/question sequence will ultimately lead to accomplishing the goal that is implied by the use-case name and description.

Since different users may have different requirements to the functionality of the system, it is also common to define
user groups that are expected to have similar types of interactions with the system. The users
within a user group typically have the same profile and role within their organization and their set of competences are expected to
be similar. This allows use cases and their associated requirements to be defined in relation to particular user groups.

In order to provide further insight into the use-case driven approach, it is useful to distinguish between functional and nonfunctional
requirements.  Functional requirements are those requirements that are directly related to the interaction between the
user and the system.  The nonfunctional requirements are hidden for the user and are more indirectly related to the  overall
success of the system.  Nonfunctional requirements can, for instance, include scalability, trace-ability and test-ability.  After use-cases are provided and functional
requirements are identified, it is the requirement engineers responsibility to identify, document and communicate these
nonfunctional requirements.  The use-case driven approach to requirement engineering thus focuses on eliciting the
functional requirements in collaboration with the users.  This process supports and improves the communication between the users and the developers,
because the focus is on what the users want and less on how the requested functionality can actually be achieved.
