\section{Conclusion, observations and reflections} 
% \fixme{For a
%   possible publication, we could report on our (including the use case providers') experiences with the process, but
%   since we are not yet finish with the RE process there is not that much to report \ldots }
\label{sec:conclusion}

%This document describes the requirements engineering process pursued in AMIDST as well as its application. The general process is  adapted and based
%on previously described approaches to requirements engineering, but tailored to the specific needs and characteristics of the AMIDST
%project. In particular, the idiosyncratic aspects of the AMIDST project that combined distinguishes it from other software
%projects at the RE level, include (i) a pre-defined project scope, (ii) many different
%stakeholders, and (iii) the development of a sufficiently general software framework that can be instantiated for
%use case providers representing different industries.  Central to the RE approach is the use case
%concept that forms the basis for the requirements specification. The actual specification is documented in a generic
%formal template that allows for the elicited requirements to be compared and prioritized across domains. 
%
%The division of work realized in the AMIDST project was partly successful due to the natural coupling (geographical and
%affinity based) between the industrial partners and the academic partners. This type of work division may not be
%achievable in projects with a larger number of partners or where the partners are not geographical co-located. On the
%other hand, it should also be emphasized that this division of work is \emph{not} as such prescribed by the proposed
%RE process. 
%
%The methodology that is outlined in the AMIDST project may transfer to other projects in two ways.  1) The other project share some of the same characteristics as the AMIDST project and some of the ideas are directly applicable. 2) The other project may have very different characteristics, but the very idea of identifying the key characteristics and stear RE choices based on those may be taken.  In this sence, the experiences we have had with the AMIDST project may serve as inspiration for other projects to come.

In this paper we have introduced a general methodology for the 
RE process in software development projects
with the following characteristics: relatively small group of
developers, a pre-defined project scope, stakeholders from
different industries, and the development of a general software
framework that can be instantiated according to the needs of different 
stakeholders.  The presented methodology adopts a use case-based
approach tailored to these specific characteristics of the project.
We also provide several key considerations for the RE process of this kind of project: division of the RE
approach in different phases to ease the overall implementation of
the process; structured prioritization of all the requirements to
ease the agreement between the stakeholders; and the employment of
a template based document to ease the elicitation of the
requirements and the communication between stakeholders with
different backgrounds, expectations, and locations.

In our opinion, the presented methodology is general enough to be 
applicable to a wide range of software development projects with 
similar characteristics. Concretely, we believe that the describe
RE process could be of great help to technology transfer based
projects between the academia and the industry.

This paper is a summary of the AMIDST projects Deliverable 1.1 \cite{Fer14}. The full report, entitled \emph{General methodology for
  requirement analysis}, also includes the general template for the developed RE process described in Section 4
and is available at the project's website \url{amidst.eu}.