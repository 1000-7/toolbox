
\section{Realization of the requirements engineering process}
\label{sec:realization}

The RE process was organized by coupling each use case provider with an
academic partner; Verdande Technology was paired with NTNU, DAIMLER was paired with AAU and Hugin, and CajaMar was paired with
UAL. The particular partner associations were based on geographical as well as affinity
considerations. It is important to stress that the RE process does \emph{not} prescribe such a
partner association, but it does bring distinct advantages. First of all, the academic partners can better assist the use case
providers when completing the requirements template, and the ongoing internal communications and discussions (both formal
and informal) provide an opportunity for early feedback on drafts of the requirements
specification. Secondly, this division of work also provides an increased knowledge transfer between industrial and academic
partners. 

% This part of the report is written in month six when most of the process is conducted.  This section contains a few points on what experiences we have gained in the project so far:

% \begin{enumerate}
% \item Everyone involved has had a learning experience on many levels.  The industrial partners have learned about probabilistic graphical models, while the academic partners have learned about the industrial domains.  Most participants have increased their knowledge on how to conduct a requirement analysis. 
% \item There has only been one meeting where all stakeholders have met, which was the kickoff meeting in Denmark in month three.  Most communication has been done through Skype and email, but also a few face to face meetings have taken place. Most of the communications have been related to clarifications in terms of filling out the template X1.
% \item There have been adjustments of the template X1 as the process has proceeded.  Examples of this is adding fields to the requirements so they could be linked to concrete tasks in the AMIDST project or adding columns for rating the importance of a requirement.
% \end{enumerate}



% \subsubsection*{Division of work}

% For each industrial partner, a mentor among the academic partners is assigned. The mentor for Verdande Technology is NTNU, the mentor for DAIMLER is AAU; and the mentor for CajaMar is UAL. Hugin has a coordination role. On one side, this allows each of the academic partners to focus on only one industrial domain.  On the other side, the industrial partners have the academic support they need for identifying proper use cases.  This division of work is a tool to mitigate characteristic two, because it eases the knowledge transfer between the industrial and academic partners. 


As described in Section~\ref{sec:AmidstRequirementProcess} one of the design considerations for the requirements
engineering process was to base the requirements specification on a formal template that would be shared by
all three use case providers. In addition to the information that the use case providers are requested to fill-in, the
template also provides a description of the overall RE process as well as guidelines on how to
complete the template. A generic template can be found in Appendix~\ref{sec:form-fram-requ}.

  
% Because of characteristics two, we have decided to not follow a RE process that heavily relies on personal meetings and
% direct personal interviews.  Even though meetings still is an important ingredient in the process, we decided to
% distribute a document template to each of the industrial partners early in the process. 

The completion of the templates was conducted as an iterative process with a close collaboration between the use case
providers and the paired academic partners. In addition to the more formal deadlines marking transitions between 
phases in the RE process, we also introduced several short-term deadlines, where
the use case providers were given feed-back on draft versions of their completed templates. Not only did this serve as an
instrument to ensure a continuous progression in the requirements specification, where misunderstandings and problems
could be identified and mitigated at an early stage, but it also provided an early transfer of knowledge from the
industrial partners to the academic partners in the project. Part of this (otherwise tacit) knowledge were documented
for the benefit of the other partners, both current and future, in the consortium, and is expected to be included 
in the deliverables planned for Work packages 6--8. This, e.g., includes a description of the data characteristics for
the use case providers. 

% \begin{itemize}
% \item Industrial partners are pushed to contribute with use cases early, meaning that issues and misunderstandings are revealed early.  This is important to mitigate characteristic three.
% \item All ideas are documented and there is no loss of information, which is common in interviews.  This is important to meet characteristic one, three, four and five.
% \item The provided information can easily be transferred from the mentors to the other academic partners.  This is important for meeting characteristic four.
% \item  The partners can spend more time on use cases and requirements, before talking to the mentors.  To some extent, this meets characteristics three, because it is easier for the mentors to learn when the requirements are more though through.
% \item The work on the different templates can be asynchronous in time.  This eases the resources allocation for the different partners.  This is related to characteristic two.
% \end{itemize}

The specified requirements (identified by a unique label as described in Appendix~\ref{sec:form-fram-requ}) together
with their work package/task allocations and prioritizations will be summarized in
tables at the work package level. These tables allow work package leaders to get a clear overview of the specific
requirements that need to be taken into account in the different work packages. An example of a part of such a work
package requirements table can be found
in Table~\ref{tab:WP2-requirements}, which includes some of the presently collected requirements pertaining to Work package 2. 


\begin{table}[htbp]
  \centering
  \begin{tabular}{|c|c|c|c|c|}
    \hline
    Req.\ ID.\ & Relevant subphase & Must/should/could & Points & Task \\ \hline\hline
    DAI.U5.D1 & Framework devel.\ \& instan.\ & Should & 20 & 2.2  \\
    DAI.U5.D2 & Framework devel.\ \& instan.\ & Should & 20 & 2.2  \\
    DAI.U5.D3 & Framework devel.\ & Should & 15 & 2.2  \\
    DAI.U5.D4 & Framework devel.\ & Should & 15 & 2.2  \\
    DAI.U5.D4 & Framework instant.\ & Should & 20 & 2.2  \\
    DAI.U7.D1 & Framework devel.\ & Must & 35 & 2.1 \\
   \vdots & \vdots  & \vdots & \vdots & \vdots \\ \hline\hline
  \end{tabular}
  
  \caption{The work package requirements table containing the presently collected requirements for Work package 2.}
  \label{tab:WP2-requirements}
\end{table}



%%% Local Variables: 
%%% mode: latex
%%% TeX-master: "REproccess"
%%% End: 
