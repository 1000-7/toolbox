
\section{Realization of the requirements engineering process}
\label{sec:realization}

This part of the report is written in month six when most of the process is conducted.  This section contains a few points on what experiences we have gained in the project so far:

\begin{enumerate}
\item Everyone involved has had a learning experience on many levels.  The industrial partners have learned about probabilistic graphical models, while the academic partners have learned about the industrial domains.  Most participants have increased their knowledge on how to conduct a requirement analysis. 
\item There has only been one meeting where all stakeholders have met, which was the kickoff meeting in Denmark in month three.  Most communication has been done through Skype and email, but also a few face to face meetings have taken place. Most of the communications have been related to clarifications in terms of filling out the template X1.
\item There have been adjustments of the template X1 as the process has proceeded.  Examples of this is adding fields to the requirements so they could be linked to concrete tasks in the AMIDST project or adding columns for rating the importance of a requirement.
\end{enumerate}



\subsubsection*{Division of work}

For each industrial partner, a mentor among the academic partners is assigned. The mentor for Verdande Technology is NTNU, the mentor for DAIMLER is AAU; and the mentor for CajaMar is UAL. Hugin has a coordination role. We considered geographical and affinity reasons for these assignments.  On one side, this allows each of the academic partners to focus on only one industrial domain.  On the other side, the industrial partners have the academic support they need for identifying proper use cases.  This division of work is a tool to mitigate characteristic two, because it eases the knowledge transfer between the industrial and academic partners. 


\subsubsection*{Document template}

Because of characteristics two, we have decided to not follow a RE process that heavily relies on personal meetings and direct personal interviews.  Even though meetings still is an important ingredient in the process, we decided to distribute a document template to each of the industrial partners early in the process. The document template contains a description of how to describe the system context that the AMIDST software shall run in.  Moreover, it contains a description of how to write the use cases, how to define the user groups and how to document the requirements.
The main advantages with this template are:

\begin{itemize}
\item Industrial partners are pushed to contribute with use cases early, meaning that issues and misunderstandings are revealed early.  This is important to mitigate characteristic three.
\item All ideas are documented and there is no loss of information, which is common in interviews.  This is important to meet characteristic one, three, four and five.
\item The provided information can easily be transferred from the mentors to the other academic partners.  This is important for meeting characteristic four.
\item  The partners can spend more time on use cases and requirements, before talking to the mentors.  To some extent, this meets characteristics three, because it is easier for the mentors to learn when the requirements are more though through.
\item The work on the different templates can be asynchronous in time.  This eases the resources allocation for the different partners.  This is related to characteristic two.
\end{itemize}


\subsubsection*{Requirements and work packages allocation}

To make sure that all requirement comply with the document of work that is agreed upon upfront, every requirement is linked with the deliveries that are relevant.  This is clearly a tool to mitigate characteristic one.

\subsubsection*{Requirements prioritization}

In the Amidst RE process, the industrial partners are asked to categorize every requirement as an either must, should or could requirement.  Moreover, they are also asked to give rigorous prioritization of each requirement.  The intention is that the industrial partners are forced to think thoroughly about every requirement and how it relates to solving their problems.  This is a way to mitigate characteristic five.

