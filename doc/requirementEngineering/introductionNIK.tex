\section{Introduction}

The requirements engineering process adopted by a particular software project typically reflects intrinsic characteristics of the project in question. Thus, there is often no standard way to conduct a RE process \cite{Poh10},  but one rather adapts existing processes to the needs and specifics of the project. This especially applies to smaller software projects \cite{Qui10} and \cite{Ara07}, where a best practice seems to be lacking, and the RE processes follow more ad-hoc strategies.

In this paper, the RE process is outlined for the project named Analysis of MassIve DataSTreams (AMIDST). AMIDST is a project that is partially funded by the European Union's Seventh Framework Programme for research, technological development and demonstration, and falls in the category of a small development project. AMIDST has a pre-specified scope, because the grant agreement asserts that the result of theRE process is in compliance with the {\em Description of Work} (DoW). The DoW defines the scope of the project as well as more general requirements to the system, both functional and non-functional, and e.g. includes a fixed deadline for the RE process. This restricts the project from, e.g., following a strict agile approach \cite{Din10},  where a product owner is continuously renegotiating the requirements and where the RE process is seen as a continuous process throughout the whole project \cite{Kav11}. Furthermore, the software framework developed in the project should be sufficiently general to accommodate stakeholders and use case providers representing different industries and not only the ones being part of the consortium. Thus, the RE process relates not only to the software framework, but also to the solutions to be developed for the use cases. Specifically, for each use case provider the general AMIDST framework will be instantiated in order to meet the needs and requirements of that use case provider.

To the best of our knowledge, there exist no guidelines for the RE process for small projects with these characteristics. This paper therefore  attempts to make a first step towards defining a best practice for such situations. The project's characteristics and challenges have formed the basis for the development of the AMIDST RE process. Based primarily on organizational characteristics of the project, the RE process is composed of selected components from existing RE processes that have been tailored to the identified AMIDST characteristics. However, we believe that the characteristics are sufficiently general to make them applicable for other small projects as well. In particular, it is expected that projects that share at least some of the characteristics of the AMIDST project may draw on the methodology that is outlined in this paper.

The paper is outlined as follows. In section  \ref{sec:stateOfArt}, the basic principles of requirement engineering are briefly outlined. Section  \ref{sec:AmidstRequirementProcess} starts by describing the main characteristics of the AMIDST project, before the RE process is outlined. In section  \ref{sec:realization}, we have described the realization of the process, and the report is concluded in section  \ref{sec:conclusion}.
