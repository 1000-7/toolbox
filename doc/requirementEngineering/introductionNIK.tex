\section{Introduction}

The requirements engineering process adopted by a particular software project typically reflects intrinsic
characteristics of the project in question. Thus, there is often no standard way to conduct a RE process \cite{Poh10}, but one rather adapts existing processes to the needs and specifics of the project. This
especially applies to smaller software projects in which category the AMIDST project belongs. To our
knowledge only two surveys have been conducted on RE processes for small software projects with less than 10 developers;
see \cite{Qui10} and \cite{Ara07}.  Based on these two studies, it is clear that RE processes in small companies follow
more ad-hoc strategies and that there seems to be little agreement on how to conduct the RE process in practice. 
 
Another defining characteristic of the AMIDST project is that there exist a DoW prior to project start.  The DoW defines the scope of the project as well as more general requirements to the system, both functional and non-functional, and e.g.\ includes a fixed deadline for the RE process. This restricts the project from, e.g., following a strict Agile approach
\cite{Din10}, where a product owner is continuously renegotiating the requirements and where the RE process is seen as a
continuous process throughout the whole project \cite{Kav11}.

Finally, the software framework developed in the project should be sufficiently general to accommodate stakeholders and
use case providers representing different industries and not only the ones being part of the consortium. Thus, the RE
process relates not only to the software framework, but also to the solutions to be developed for the use
cases. Specifically, for each use case provider the general AMIDST framework will be instantiated in order to meet the
needs and requirements of that use case provider.    

These characteristics and challenges are usually associated with larger software projects, and have  formed
the basis for the development of the AMIDST RE process: Based on primarily organizational characteristics of the AMIDST
project, the AMIDST RE process is composed of selected components from existing RE processes that have been tailored to the identified AMIDST characteristics. The characteristics are sufficiently general to make them
applicable for other small projects as well.  In particular, it is expected that projects that share at least some of the characteristics of the AMIDST project may draw on the methodology that is outlined in this paper.

The report is outlined as follows.  In section \ref{sec:stateOfArt}, the basic principles of requirement engineering are
briefly outlined.  Section \ref{sec:AmidstRequirementProcess} starts by describing the main characteristics of the
AMIDST project, before the RE process is outlined.  In section \ref{sec:realization}, we have described the realization
of the process, and the report is concluded in section \ref{sec:conclusion}. 
