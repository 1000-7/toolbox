\section{Introduction}

As projects differ in various ways, there is no common way to conduct the requirement engineering RE process \cite{Poh10}.  It is therefore a reasonable strategy to search the literature for a RE process that is closely related to the project at hand.  

In the AMIDST project, a first observation is that the project is very small in terms of man months assigned to development.  Despite the fact that the field of requirement engineering RE has existed since the seventies, to our knowledge only two surveys have been conducted on RE for very small software projects with less than 10 developers; \cite{Qui10} and \cite{Ara07}.  Based on these two studies, it is clear that RE processes in very small companies are more ad-hoc and that there seem to be little agreement on how to conduct the RE process in practice.

A second observation is that there exist a description of work that is agreed on before the RE process is started and that there exist an explicit date when the RE process should be finished.  In practice, this limits the project from following a strict Agile approach \cite{Din10}.  In the Agile approach, a product owner is continuously renegotiating the requirements and the RE process is seen as a continuous process throughout the whole project.  Paper \cite{Kav11} outlines how to incorporate requirement gathering in small Agile projects.  

A third observation is that there is huge widespread of stakeholders in the AMIDST project and that the software is required to interface with three different softwares from three different companies.  These are challenges that are usually associated with larger software projects.  This motivates looking into the more general literature on RE processes, rather than the literature on very small projects. 

Based on these three observations, we realized that finding a ready-to-use strategy for the AMIDST project would be very difficult.  We therefore decided to take this one step further and identify more characteristics to the AMIDST project and motivate our choices related to the RE process based on these characteristics. 

The report is outlined as follows.  In section \ref{sec:stateOfArt}, the basic principles of requirement engineering are briefly outlined.  Section \ref{sec:AmidstRequirementProcess} starts by describing the main characteristics of the AMIDST project, before the RE process is outlined.  In section \ref{sec:realization}, we have described the realization of the process so far, before the report is concluded in section \ref{sec:conclusion}.
