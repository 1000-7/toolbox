\section{Conclusion, observations and reflections} 
% \fixme{For a
%   possible publication, we could report on our (including the use case providers') experiences with the process, but
%   since we are not yet finish with the RE process there is not that much to report \ldots }
\label{sec:conclusion}

This document describes the requirements engineering process pursued in AMIDST as well as its application. The general process is  adapted and based
on previously described approaches to requirements engineering, but tailored to the specific needs and characteristics of the AMIDST
project. In particular, the idiosyncratic aspects of the AMIDST project that combined distinguishes it from other software
projects at the RE level, include (i) a pre-defined project scope, (ii) many different
stakeholders, and (iii) the development of a sufficiently general software framework that can be instantiated for
use case providers representing different industries.  Central to the RE approach is the use case
concept that forms the basis for the requirements specification. The actual specification is documented in a generic
formal template that allows for the elicited requirements to be compared and prioritized across domains. 

The division of work realized in the AMIDST project was partly successful due to the natural coupling (geographical and
affinity based) between the industrial partners and the academic partners. This type of work division may not be
achievable in projects with a larger number of partners or where the partners are not geographical co-located. On the
other hand, it should also be emphasized that this division of work is \emph{not} as such prescribed by the proposed
RE process. 
