There is a general lack of formal and agreed upon methods for conducting requirement engineering (RE) for smaller software projects \cite{Ara07,Qui10}.  Many small projects have a pre-specified scope, since the pre-specified scope often appears when the project needs to agree with a grant agreement that partially funds the project.  To our knowledge, there exist no publications on small projects where the scope is specified before the project is started.  

In this paper, the RE process is outlined for the project named: Analysis of MassIve DataSTreams (AMIDST).  AMIDST is a project that is partially funded by the European Union's Seventh Framework Programme for research, technological development and demonstration.  This project has a pre-specified scope, because the grant agreement demands that a requirement engineering process is conducted within a specified time frame and also that the result of the requirement process is in compliance with the Description of Work (DoW) that is agreed upon.   

The AMIDST RE process is based on selected methodological approaches from existing RE processes, which subsequently have been tailored to the specific characteristics of the AMIDST project.  The characteristics in the AMIDST project are; there exist a pre-specified scope, partners are located far apart, massive knowledge transfer between parners is expected, the software framework need to be applicable in three different industrial domains and that it is expected that the project focus will be refined.  

In general, a use-case driven approach is followed, because functional requirements are in focus. Another important tool in the AMIDST RE process is the development of a unified and formal template for elicitation to encourage the overall transparency of the process.