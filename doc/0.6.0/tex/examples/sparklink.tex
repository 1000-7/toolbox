\documentclass[10pt,a4paper]{article}
\usepackage[latin1]{inputenc}
\usepackage{amsmath}
\usepackage{amsfonts}
\usepackage{amssymb}
\usepackage{graphicx}
%%% formatting the code
\usepackage{listings}



\usepackage{color}
\lstset{%
	escapeinside={(*}{*)},%
}

\newcommand{\amidstversion}{\input{../../version.txt}}

\lstset{
	frameround=fttt,
%	language=java,
	numbers=left,
	breaklines=true,
	mathescape, 
	columns=fullflexible, 
	basicstyle=\fontfamily{lmvtt}\selectfont,
	keywordstyle=\color{blue}\fontfamily{lmvtt}\selectfont, 
	numberstyle=\color{black}
}
\lstMakeShortInline[columns=fixed]|



\newcommand{\includejavasource}[1]{\lstinputlisting[language=java]{#1}}
\newcommand{\inlinejava}[1]{\lstinline[columns=fixed,language=java]{#1}}

\newcommand{\lang}[1]{}



\usepackage{hyperref}

\begin{document}



\section{Sparklink: Code Examples}\label{sec:sparklink}

%\begin{itemize}
%	\item \hyperref[sec:flinklink:io]{Input/output}
%	\begin{itemize}
%		\item \hyperref[sec:flinklink:io:read]{Reading data}
%		\item \hyperref[sec:flinklink:io:write]{Write data}
%	\end{itemize}
%		\item \hyperref[sec:flinklink:learning]{Parametric Learning}
%
%		
%	\begin{itemize}
%		\item \hyperref[sec:flinklink:learning:pml]{Parallel Maximum Likelihood}
%		\item \hyperref[sec:flinklink:learning:dvmp]{Distributed Variational Message Pasing}
%		\item \hyperref[sec:flinklink:learning:dvi]{Distributed VI}
%		\item \hyperref[sec:flinklink:learning:svi]{Stochastic VI}
%	\end{itemize}
%	\item \hyperref[sec:flinklink:ext]{ Extensions and applications}
%	\begin{itemize}
%			\item \hyperref[sec:flinklink:ext:models]{ Latent variable models with Flink}
%			\item \hyperref[sec:flinklink:ext:conceptdrift]{Concept drift}
%	\end{itemize}
%\end{itemize}


\subsection{Input/output}\label{sec:sparklink:io}




\subsubsection{Reading data}\label{sec:sparklink:io:read}

In this example we show how can we read a dataset using sparklink. This module supports reading files in formats \textit{json} and \textit{parquet}. Note that the method \textit{DataSparkLoader::open} automatically detects the format of the file (the indicated path should contain the extension). Finally all the instances in the data set are printed.


\includejavasource{../../../../examples/src/main/java/eu/amidst/sparklink/examples/io/DataStreamLoaderExample.java}


\subsubsection{Writing data}\label{sec:sparklink:io:write}

Here we show how can we save spark data into a file. First a random data set is generated using the method \textit{DataSetGenerator::generate}. Afterwards, such data is save using the method \textit{DataSparkWriter::writeDataToFolder}

\includejavasource{../../../../examples/src/main/java/eu/amidst/sparklink/examples/io/DataStreamWriterExample.java}

\subsection{Parameter learning}\label{sec:sparklink:learning}

AMIDST provides parameter learning using spark with the \textit{Maximum Likelihood} algorithm. In the following example, we load a data set in format json and we use it for learning a simple naive bayes (more complex DAGs can also be learnt).

\includejavasource{../../../../examples/src/main/java/eu/amidst/sparklink/examples/learning/MaximumLikelihoodLearningExample.java}




\end{document}