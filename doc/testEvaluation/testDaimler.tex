\documentclass{article}

\usepackage{times}
\usepackage{graphicx}
\usepackage{latexsym}

\usepackage{bm}
\usepackage{amsbsy}
\usepackage{amsmath}
\usepackage{amsfonts}
\usepackage{amssymb}

\usepackage{subfigure}

\usepackage{theorem}

\theoremstyle{theorem}
\newtheorem{theorem}{Theorem}

\theoremstyle{definition}
\newtheorem{definition}{Definition}
\newtheorem{remark}{Remark}

\newcommand{\bu}[1]{\mathbf{#1}}
\newcommand{\bv}[1]{\bm{#1}}




\title{Practical Considerations for Testing the Daimler Use Case}
%\author{Sigve Hovda \\
%Norwegian University of Science and Technology\\
%Department of Computer and Information Science,
%Trondheim, Norway\\
%sigveh@idi.ntnu.no}
\date{}


\begin{document}
\maketitle

There are two application scenarios here.  The first one is early recognition of lane change manoeuvre. The second is prediction of the need for lane change based on relative dynamics between two vehicles following the same lane.

\section{Early recognition of lane change manoeuvre}

It is our understanding that a \emph{lane change manoeuvre} is either a EGO cut in, EGO cutout, object cut in and object cutout.  Moreover, it is also our understanding that a \emph{no lane change manoeuvre} is either a object follow or a lane follow.  In the testing regime we have a binary classification problem: lane change or no lane change and the subdivision  into the other categories is not a concern regarding testing this particular application scenario.

Daimler has provided us with a number of time series of approximately 5 seconds each.  The resolution is 42 ms between each sample, but sometimes there are missing samples.  Commonly there are no more than 84 ms between each measurement.  There are 93 time series of lane follow and 148 time series of lane change.  The lane change happens always on the last sample. There are no overlap between any of these time series, and they can be seen as independent of each other.

The system shall run on these time series and provide us with information about if there is a lane change and if there is a lane change, how many seconds before the actual crossing did it happen.

The number of true negatives TN is defined as the number of times the system reported lane follow when there actually was a lane follow. 

There are two definitions of the number of true positives TP1 and TP2.  TP1 is the number of times the system reported a lane crossing more than one second before the lane crossing and TP2 is the number of times the system reported a lane crossing more than two second before the lane crossing.

\subsection*{Requirements}
Two requirements are needed.  For the first use case scenario it is required that AUROC should be above 0.96 for prediction 1 second before lane crossing and 0.90 for the 2 second prediction.  We will refer to these as AUROC1 and AUROC2.  

AUROC1 is calculated from TN and TP1 and the total number of positives and negatives for various threshold settings.  AUROC2 is calculated from TN and TP2 and the total number of positives and negatives for various threshold settings.  

\subsection*{Questions: }
\begin{enumerate}
\item Can you go though all the information and make sure that it is correct.
\end{enumerate}


\section{Recognition of need for lane change manoeuvre}

It is our understanding that Daimler has not yet provided us with data for this use case scenario.  In order to test this use case scenario, we need a set of time series where there is no need for lane change and a data set where it is actually a need for lane change. 

\subsection*{Questions: }
\begin{enumerate}
\item Should the time series for no need for change be time series of object follow where no lane change happens?
\item Should the time series for need for change be time series of object follow where there is a lane change at the end?
\item If the answer is yes to both questions, can we follow a similar path to what with did in first use case scenario?
\end{enumerate}

\bibliographystyle{named}
\bibliography{ijcai13}

\end{document}

